% JUMP TO LINE 60, 75
\documentclass{article}
\usepackage[letterpaper,portrait,top=0.4in, left=0.6in, right=0.6in, bottom=1in]{geometry}

\usepackage{amsmath, amsfonts, amsthm, amssymb}
\usepackage{graphicx, float}
\usepackage{suffix}
\usepackage{multicol}
\usepackage{cancel}
\usepackage{mdframed}
\usepackage{mathtools}
\usepackage{tcolorbox}
\usepackage{hyperref}
\usepackage[per-mode=symbol]{siunitx}
\usepackage{setspace}
\usepackage{parskip}
\usepackage{titling}
\usepackage{mlmodern}

\newcommand{\alignedintertext}[1]{%
  \noalign{%
    \vtop{\hsize=\linewidth#1\par
    \expandafter}%
    \expandafter\prevdepth\the\prevdepth
  }%
}

\newcommand{\definition}[1]{\begin{tcolorbox}[colback=red!5!white,colframe=red!75!black,parbox=false] #1 \end{tcolorbox}}
\newcommand{\theorem}[2]{\begin{tcolorbox}[title={#1},colback=blue!5!white,colframe=blue!75!black,parbox=false] #2 \end{tcolorbox}}
\WithSuffix\newcommand\theorem*[1]{\begin{tcolorbox}[colback=blue!5!white,colframe=blue!75!black,parbox=false] #1 \end{tcolorbox}}

\title{\vspace*{-40pt}AP Physics C -- Class Notes}
\author{Jayden Li}
\date{\today}

\begin{document}
\setstretch{1.25}
\fontsize{11pt}{12pt}\selectfont
\setlength{\abovedisplayskip}{\abovedisplayskip/2}
\setlength{\belowdisplayskip}{\belowdisplayskip/2}
\setlength{\parindent}{0pt}
\setlength{\parskip}{2ex plus 0.5ex minus 0.2ex}
\maketitle

\tableofcontents
% \newpage

\section{Introduction}

\subsection{Jumping Monsters}

See Figure 1.1 in Notebook.

We investigate the relationship between the mass of the toy $m$ and the change in height $\Delta h$. Equipment:
\begin{itemize}
	\item Meter stick (not ruler, since ruler is only 30cm long)
	\item Phone (to record video)
	\item Balance (to measure mass in grams and kilograms, a scale measures weight in Newtons)
	\item Washers, paper clips and tape (to increase mass of toy)
\end{itemize}

We collect many data points. We will collect 5 data points, which is 5 conditions, which is 5 different masses to test.  We want to repeat every mass a few times too; we will test every mass 3 times (``3 trials''). In total, the toy will jump $5\cdot 3=15$ times. Trial means that conditions/masses are the same.

Results/data are in Table 1.2 in Notebook.

Based on conservation of energy:
\begin{equation*}
    \text{PE}_ \text{s}= \text{PE}_ \text{g}
	\implies \frac12kx^2=mgh
	\implies h=\frac{kx^2}{2mg}=\frac{kx^2}{2g}\cdot\frac1m
\end{equation*}
If we graph mass $m$ against height $\Delta h$, this is an inverse relationship, as $kx^2/2g$ is a constant (the spring distance $x$ does not change for one toy, $k$ is spring constant, and $g$ is acceleration due to gravity).

Because we want a linear relationship, we can graph inverse mass $1/m$ against height $\Delta h$. This becomes a line with slope $kx^2/2g$.

\end{document}
