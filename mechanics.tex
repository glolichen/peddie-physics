\documentclass{article}
\usepackage[letterpaper,portrait,top=0.4in, left=0.6in, right=0.6in, bottom=1in]{geometry}

\usepackage{amsmath, amsfonts, amsthm, amssymb}
\usepackage{soul}
\usepackage{graphicx, float}
\usepackage{suffix}
\usepackage{soul}
\usepackage{esdiff}
\usepackage{multicol}
\usepackage{cancel}
\usepackage{mdframed}
\usepackage{mathtools}
\usepackage{tcolorbox}
\usepackage[colorlinks, linkcolor=blue]{hyperref}
\usepackage[per-mode=symbol]{siunitx}
\usepackage{setspace}
\usepackage{parskip}
\usepackage{enumitem}
\usepackage{titling}
\usepackage{mlmodern}

\newcommand{\alignedintertext}[1]{%
  \noalign{%
    \vtop{\hsize=\linewidth#1\par
    \expandafter}%
    \expandafter\prevdepth\the\prevdepth
  }%
}

\newcommand{\definition}[1]{\begin{tcolorbox}[colback=red!5!white,colframe=red!75!black,parbox=false] #1 \end{tcolorbox}}

\newcommand{\theorem}[2]{\begin{tcolorbox}[title={#1},colback=blue!5!white,colframe=blue!75!black,parbox=false] #2 \end{tcolorbox}}
\WithSuffix\newcommand\theorem*[1]{\begin{tcolorbox}[colback=blue!5!white,colframe=blue!75!black,parbox=false] #1 \end{tcolorbox}}

\newcommand{\example}[2]{\begin{tcolorbox}[title={Example: #1},colback=brown!5!white,colframe=brown!75!black,parbox=false] #2 \end{tcolorbox}}

\newcommand{\remark}[2]{\begin{tcolorbox}[title={#1},colback=black!5!white,colframe=black!75!black,parbox=false] #2 \end{tcolorbox}}
\WithSuffix\newcommand\remark*[1]{\begin{tcolorbox}[colback=black!5!white,colframe=black!75!black,parbox=false] #1 \end{tcolorbox}}

\newcommand{\hlc}[2][yellow]{\sethlcolor{#1}\hl{#2}}

\newcommand*{\deriv}[1][x]{\ensuremath{\dfrac{\mathrm{d}}{\mathrm{d}#1}}}
\newcommand*{\floor}[1]{\ensuremath{\lfloor #1\rfloor}}

\title{\vspace*{-40pt}\textsc{AP Physics C: Mechanics Notes}}
\author{Jayden Li}
\date{Fall and Winter Trimesters, 2025}

\begin{document}
\setstretch{1.25}
\fontsize{11pt}{12pt}\selectfont
\setlength{\abovedisplayskip}{\abovedisplayskip/2}
\setlength{\belowdisplayskip}{\belowdisplayskip/2}
\setlength{\parindent}{0pt}
\setlength{\parskip}{2ex plus 0.5ex minus 0.2ex}
\maketitle

\tableofcontents
% \newpage

\section{Introduction}

\subsection{Jumping Monsters}

See Figure 1.1 in Notebook.

We investigate the relationship between the mass of the toy $m$ and the change in height $\Delta h$. Equipment:
\begin{itemize}
	\item Meter stick (not ruler, since ruler is only 30cm long)
	\item Phone (to record video)
	\item Balance (to measure mass in grams and kilograms, a scale measures weight in Newtons)
	\item Washers, paper clips and tape (to increase mass of toy)
\end{itemize}

We collect many data points. We will collect 5 data points, which is 5 conditions, which is 5 different masses to test.  We want to repeat every mass a few times too; we will test every mass 3 times (``3 trials''). In total, the toy will jump $5\cdot 3=15$ times. Trial means that conditions/masses are the same.

Results/data are in Table 1.2 in Notebook.

Based on conservation of energy:
\begin{equation*}
    \text{PE}_ \text{s}= \text{PE}_ \text{g}
	\implies \frac12kx^2=mgh
	\implies h=\frac{kx^2}{2mg}=\frac{kx^2}{2g}\cdot\frac1m
\end{equation*}
If we graph mass $m$ against height $\Delta h$, this is an inverse relationship, as $kx^2/2g$ is a constant (the spring distance $x$ does not change for one toy, $k$ is spring constant, and $g$ is acceleration due to gravity).

Because we want a linear relationship, we can graph inverse mass $1/m$ against height $\Delta h$. This becomes a line with slope $kx^2/2g$.

Logger pro graph, plotting inverse mass in $1/\text{kg}$ against height in meters:

\includegraphics[width=\linewidth]{class/loggerpro0903.png}
\begin{equation*}
	\text{Slope}
	=m 
	=\frac{kx^2}{2g}
	=0.004880
	\implies k=m\cdot \frac{2g}{x^2}
	=0.004880\cdot \frac{2(9.81)}{(1.5/100)^2}
	=\boxed{425.536\,\si{\newton\per\meter}}
\end{equation*}

\remark{Linearization}{Linearization is a powerful technique. For example, with Kepler's third law of planetary motion:
\begin{equation*}
    \frac{T^2}{r^3}=\frac{4\pi^2}{GM}
\end{equation*}
is annoying to plot when we plot $T$ against $r$. We rearrange and plot $T^2$ against $r^3$ for a linear relationship:
\begin{equation*}
    T^2=\frac{4\pi^2}{GM}r^3
\end{equation*}
$4\pi^2/GM$ is a constant/the slope. We take $y=T^2$ and $x=r^3$, and we have the fitted slope $m=4\pi^2/GM$.}

\subsection{Projectile Motion}

Two-dimensional motion, $x$ and $y$-directions. The type of motion in the two directions are different.

$x$-direction has constant velocity because there is no net force and no acceleration acting in the $x$-direction.
\begin{equation*}
    v_x=\frac{\Delta x}{\Delta t}
	\iff \Delta x=v_x(\Delta t)
\end{equation*}

$y$-direction has constant acceleration due to gravity $-g$.
\begin{align*}
	v_y&=v_{0y}+at \\
	\Delta y&=v_{0y}t+\frac12at^2 \\
	v_y^2&=v+{0y}^2+2a\Delta y
\end{align*}

\subsubsection{Special cases}

In horizontal projectile motion, $v_{0y}=0\,\si{\meter\per\second}$.

When the projectile reaches its highest point in general 2-dimensional motion, then at the highest point, there is no $y$-velocity: $v_y=0\,\si{\meter\per\second}$.

If the projectile is launched at a certain velocity $v_0$ at an angle $\theta$, we have initial $x$ and initial $y$-velocities:
\begin{align*}
	v_x=v_{0x}&=v_0\cos\theta \\
	v_{0y}&=v_0\sin\theta
\end{align*}

\subsection{Momentum Lab}

Equipment:
\begin{itemize}
	\item Motion sensor/detector
	\item Force sensor: measures force
	\item Cart connected to plunger; plunger is a metal rod with a spring inside
\end{itemize}

Push cart on a track, measure force and velocity to obtain the mass of the cart. To improve the accuracy and lower the percent error:
\begin{itemize}
	\item Check if the track is level, if the track is inclined then the cart is not moving at a constant velocity (disregarding friction) before and after the collision.
	\item Reduce friction
	\item Increase the number of \textbf{experimental conditions} (cannot increase the number of trials, because trial has the same conditions like initial speed, which is not possible to make sure) by pushing it more times
	\item Line up force sensor and cart/track
\end{itemize}

\subsection{AP Formula Sheet}

$G$ universal gravitational constant is important. We should take acceleration due to gravity on Earth $g=10\,\si{\meter\per\second\squared}$. Magnitude of gravitational field of earth is also $g=10\,\si{\newton\per\kilo\gram}$. These units are equivalent.

Kinematic equations for \textbf{constant acceleration} and therefore constant force:
\begin{equation*}
    \Delta x=v_0t+\frac12at^2 \qquad
	v=v_0+at \qquad
	v^2=v_0^2+2a\Delta x
\end{equation*}
Newton's second law does not have to be complicated; $\Sigma F=ma$ is fine.

Maximum friction $\lVert \vec{F}_f \rVert =\lVert \mu \vec{F}_N \rVert $ tells us, where $\vec{F}_N$ is normal force:
\begin{itemize}
	\item kinetic friction $\vec{F}_k=\mu_k \vec{F}_N$
	\item Maximum static friction $\vec{F}_\text{s,max}=\mu_\text{s,max}\vec{F}_N$
	\item Static friction $\lVert \vec{F}_\text{s} \rVert <\vec{F}_\text{s,max}$.
\end{itemize}

\section{Calculus}
\label{sec:calculus}

\subsection{Calculus-Based Equations}

\theorem{Kinematics}{
	\begin{alignat*}{2}
		v&=\diff xt &\qquad \Delta x=\int v\,\mathrm{d}t \\
		a&=\diff vt=\diff[2]{x}{t} &\qquad \Delta v=\int a\,\mathrm{d}t
	\end{alignat*}
	where $x$ is position, $v$ is velocity and $a$ is acceleration. These will apply to all situations (for constant velocity, constant acceleration or nonconstant acceleration).
}

Integrate with respect to independent variable; integrand is dependent variable. See Figure 2.1.

\theorem{Connecting dynamics/forces and kinematics}{
Newton's second law: $\Sigma F=ma$. Acceleration $a$ connects forces and dynamics to kinematics.

Suppose we are given position $x$. Then we can calculate the acceleration by differentiation:
\begin{equation*}
	a=\diff[2]xt
	\implies F=ma=m\diff[2]xt
\end{equation*}
}

Connecting energy and kinematics by kinetic energy $K=mv^2/2$.

\theorem{Work and force-position graphs}{
	Consider a force-position graph (Figure 2.2). The area under the curve is the work done, or change in energy: $W=\Delta E$. The relationship is as follows, where $\Sigma F$ is the net force:
	\begin{equation*}
	    W=\Delta E=\int\Sigma  F\,\mathrm{d}x
		\qquad \Sigma F=\diff Ex
	\end{equation*}
}

\theorem{Power and time}{
	Power is the rate of change of energy with respect to time:
	\begin{equation*}
		\Delta E=\int P\,\mathrm{d}t \qquad P=\diff Et
	\end{equation*}
	The change in energy is the area under the curve of a power-time graph (Figure 3.3).
}

\theorem{Force and time}{
	Recall that for constant net force $\Sigma F$, impulse $J=\Delta p=\Sigma F\cdot \Delta t$. For a force-time graph in Figure 2.4:
	\begin{equation*}
	    J=\Delta p=\int F\,\mathrm{d}t
		\qquad 
		F=\diff{p}{t}=m\cdot \diff vt=ma
	\end{equation*}
	(Impulse $\Delta p=I=J=\Delta(mv)=m \cdot \Delta v$, since mass usually does not change. $\mathrm{d}p=F\mathrm{d} t \implies \mathrm{d}(mv)=m\mathrm{d} v=F\mathrm{d} t\implies F=m(\mathrm{d}v/\mathrm{d}t)=ma$)
}

\theorem{Rotational kinematics}{
	\begin{equation*}
		\omega=\diff{\theta}t \qquad
		\alpha=\diff{\omega}{t} \qquad
		\Delta\theta=\int \omega\,\mathrm{d}t \qquad
		\Delta \omega=\int \alpha\,\mathrm{d}t
	\end{equation*}
}

\subsection{Potential Energy}

Suppose we drop a ball from a certain height. We know that the work done by the force of gravity is positive work, but since energy is conserved, the potential energy in the ball must decrease. Therefore, we have:
\begin{equation*}
    -W=\Delta U 
\end{equation*}
But, we know that work is force times distance:
\begin{equation*}
	\Delta U=-F_y\cdot \Delta y
	\implies F_y=-\frac{\Delta U}{\Delta y}
\end{equation*}
where $\Delta y$ is the distance the ball is dropped, or the distance over which the force acts. By black magic, we can change this into a derivative:
\begin{equation*}
	F_y=-\diff Uy
\end{equation*}
Must have negative sign. So, we can also have potential energy in terms of force:
\begin{equation*}
    F_y=-\diff Uy
	\implies -F_y \,\mathrm{d}y=\mathrm{d}U
	\implies \Delta U=-\int F_y\,\mathrm{d}y
\end{equation*}
\remark*{In nature, objects want to lower their potential energy. (NOTE: lower the number, \ul{NOT} the magnitude of kinetic energy.)}

\theorem{Potential energy}{
	The relationship between potential energy and force is, where $x$ is position:
	\begin{equation*}
	    \Delta U_x=-\int F_x\,\mathrm{d}x
		\qquad
		F_x=-\diff Ux
	\end{equation*}
}

In simple harmonic motion, the potential energy-displacement from equilibrium graph is a positive parabola passing through the origin. There is no potential energy at equilibrium ($x=0$) and \ul{potential energy is maximized at the highest displacement.}

Total energy is the sum of kinetic energy and potential energy: $K+U$. By conservation of energy, the total energy is constant for all displacement.

\definition{The work done by a \textbf{conservative force} only depends on the initial and final position of the object, not on the path taken.}

A lot of forces are path-dependent. For example, friction depends on the length of the path, so it is not conservative. \ul{Forces associated with change in potential energy are conservative.}

\example{Gravitational potential}{
	Change in gravitational potential energy is $\Delta U_g=mgy$. We can calculate the force:
	\begin{equation*}
		F_g=-\diff{U_g}{y}=-\deriv[y](mgy)=-mg
	\end{equation*}
	is consistent with the result we get from $F=ma$.
}

\example{Spring potential}{
	Change in spring potential is $U_s=kx^2/2$:
	\begin{equation*}
		F_s=-\diff{U_s}{x}=-\deriv[x]\left(\frac12kx^2\right)=-kx
	\end{equation*}
}

\subsection{Springs}

Formulas like $F_s=-kx$ and $U_s=kx^2/2$ depend on a \textbf{perfect, ideal spring}, which requires some assumptions.
\begin{itemize}
	\item The mass of the spring itself is negligible; the mass of the system is exactly the mass of the object attached to the spring.
	\item Force exerted is $F_s=-kx$: force-displacement graph is linear and passes through the origin. From this, formula for kinetic energy follows: $U_s=-\int F_s\,\mathrm{d}x=-\int -kx\,\mathrm{d}x=kx^2/2$.
	\item No internal friction and losses. If there are, \textit{damping} occurs. This is drawn in Figure 2.5. Even though the period does not change, each successive crest and trough decreases in magnitude.
\end{itemize}

Figure 2.6 shows the energy position graph of an ideal spring. By conservation of energy, total energy $\text{TE}=U_s+K$.

Real springs are non-ideal.

\section{Conservation of Momentum}
\label{sec:momentum}

\subsection{Explosions}

\definition{An \textbf{explosion} is when two objects separate after being together.}

Cart of mass $m_1$ and mass $m_2$ separate after a compressed spring between them, and they move at velocity $v_1$ and $v_2$. $m_1>m_2$. By conservation of momentum:
\begin{equation*}
    m_1v_1=m_2v_2
	\implies v_2=\frac{m_1v_1}{m_2}
\end{equation*}
We calculate each object's kinetic energy:
\begin{align*}
	K_1&=\frac12m_1v_1^2 \\
	K_2&=\frac12m_2v_2^2
	   =\frac12m_2 \left( \frac{m_1v_1}{m_2} \right)^2
	   =\frac12 m_1\left(\frac{m_1}{m_2}\right)v_1^2
	   =\frac{m_1}{m_2}K_1
\end{align*}
Since $m_1>m_2$: $m_1/m_2>1\implies K_2>K_1$

\theorem*{If momentum is conserved after a collision or explosion, then the object with the lower mass (and therefore higher velocity) has higher kinetic energy.}

\subsection{Elastic Head-On Collisions}

\definition{In a \textbf{head-on collisions}, both objects are moving in a straight line. In a \textbf{glancing collision}, objects will go in different angles after the collision.}

Suppose that two objects collide head on. The objects are initially traveling at velocities $v_1,v_2$, and have velocities $v_1',v_2'$ after the collision.

By conservation of momentum and conservation of kinetic energy (because collision is elastic):
\begin{align*}
	\Sigma p&=\Sigma p' \\
	m_1v_1+m_2v_2&=m_1(v_1')+m_2(v_2') \tag{1} \\
	\frac12m_1v_1^2+\frac12m_2v_2^2&=\frac12m_1(v_1')^2+\frac12m_2(v_2')^2 \tag{2}
\end{align*}
We can use this to derive another equation:
\begin{align*}
	(2)
	&\implies \frac12m_1v_1^2+\frac12m_2v_2^2=\frac12m_1(v_1')^2+\frac12m_2(v_2')^2
	\implies m_1v_1^2+m_2v_2^2=m_1(v_1')^2+m_2(v_2')^2 \\
	&\implies m_1 v_1^2-m_1(v_1')^2=m_2(v_2')^2-m_2v_2^2
	\implies m_1 \left( v_1^2-(v_1')^2 \right)=m_2 \left( (v_2')^2-v_2^2 \right) \\
	&\implies m_1(v_1+v_1')(v_1-v_1')=m_2(v_2'+v_2)(v_2'-v_2) \\
	(1)
	&\implies m_1v_1-m_1(v_1')=m_2(v_2')-m_2v_2
	\implies m_1 \left( v_1-v_1' \right)=m_2 \left( v_2'-v_2 \right) \\
	\frac{(2)}{(1)}
	&\implies \frac{\bcancel{m_1}(v_1+v_1')(\cancel{v_1-v_1'})}{\bcancel{m_1}(\cancel{v_1-v_1'})}=\frac{\bcancel{m_2}(v_2'+v_2)(\cancel{v_2'-v_2})}{\bcancel{m_2}(\cancel{v_2'-v_2})}
	\implies v_1+v_1'=v_2+v_2'
\end{align*}

\theorem{Equations for a head-on elastic collision}{
	\begin{align*}
		m_1v_1+m_2v_2&=m_1(v_1')+m_2(v_2') \tag{Conservation of Momentum} \\
		\frac12m_1v_1^2+\frac12m_2v_2^2&=\frac12m_1(v_1')^2+\frac12m_2(v_2')^2 \tag{Conservation of Kinetic Energy} \\
		v_1+v_1'&=v_2+v_2'
	\end{align*}
}

\subsection{Ballistic Pendulum}

Bullet of mass $m$ into a block of mass $M$ at velocity $v$. The bullet lodges into the mass and travels with the block. This is a completely inelastic collision.

We can calculate the final velocity of the block by conservation of momentum $mv=(M+m)V$. From $V$ we can calculate the kinetic energy $K$, and gravitational potential energy when the block is shot is $0$. So the total mechanical energy is $K+0=K$.

The highest point in the pendulum is reached when the gravitational potential energy equals the total energy:
\begin{equation*}
	\frac12(\cancel{M+m})V^2=(\cancel{M+m})g\Delta h
	\implies \Delta h=\frac{V^2}{2g}
\end{equation*}

\subsection*{Free Response Questions}

\remark{Free response}{\begin{itemize}
	\item Use pencil or black/blue pen.
	\item Start by writing known equation.
	\item Substitution (substitute numbers with no units).
	\item Final answer clearly indicated such as \boxed{\text{boxed}}.
		\begin{itemize}
			\item If numerical, do not keep infinite number of significant figures: no square root, fraction, constants; \textbf{always 2 or 3 significant figures, never 1 s.f.!!!!!}
			\item If symbolic, leave square roots, fractions, constants:
		\end{itemize}
	\item Graphing: label axes with units:
		\begin{itemize}
			\item Name of graph is ``[value on $y$ axis] vs [value on $x$ axis]'' or ``[value on $y$] wrt [value on $x$]''.
			\item Line of best fit: equal number of points above and below the line.
			\item When determining slope of line of best fit, do not choose original values: $\text{Slope}=\Delta y/\Delta x=(y_1-y_0)/(x_1-x_0)$. \ul{The slope has units.}
		\end{itemize}
	\item Integrals: must have limits of integration and differential.
	\item \ul{All quantities have units.}
\end{itemize}}

\subsection{Equilibrium}

\begin{itemize}
	\item In stable equilibrium, when a small external force is exerted, the object will move, and when the external force is removed, the system will return to its equilibrium position.
	\item In unstable equilibrium, the system will not return to equilibrium when the external force is removed.
	\item In neutral equilibrium, the object remains at the new location after the external force is removed.
\end{itemize}

\section{Center of Mass}
\label{sec:centerofmass}

\definition{The \textbf{center of mass} of a system is the average location of mass of a system or object. If the system/object is in a gravitational field then this is also called the center of gravity because we consider that is where the force of gravity/weight acts on the system.}

\begin{itemize}
	\item Object will rotate around its center of mass if force is applied
	\item If a normal force/support is applied at its center of mass, it will balance.
	\item The center of mass will change if the object's mass distribution changes.
	\item A system can be modeled as a single object located at the system's center of mass.
	\item If an object is symmetrical, then the center of mass is on a line of symmetry.
\end{itemize}

\theorem{Center of mass of system of particles}{The center of mass of a system of particles $( x_\text{cm}, y_\text{cm})$, each particle having mass $m_i$ and location $(x_i,y_1)$ is:
\begin{equation*}
     x_\text{cm}=\frac{\sum m_ix_i}{\sum m_i}
	\qquad  y_\text{cm}=\frac{\sum m_i y_i}{\sum m_i}
\end{equation*}
If we are asked to calculate the center of mass of a shape, we can try to reduce the shape into particles, and use these equations. These equations for $ x_\text{cm}, y_\text{cm}$ can be extended to any number of dimensions.}

As with torque, the reference point from which positions $x_i$ are measured is completely arbitrary.

The velocity of the whole system (from the center of mass position) is:
\begin{equation*}
	 v_\text{cm}
	=\diff{ x_\text{cm}}{t}
	=\deriv[t] \frac{\sum m_ix_i}{\sum m_i}
	=\frac{\sum m_i \diff{x_i}{t}}{M}
	=\boxed{\frac{\sum m_i v_i}{M}}
	\implies M v_\text{cm}=\boxed{p_\text{cm}=\sum m_iv_i} \\
\end{equation*}
The velocity of the center of mass is a weighted average of the velocities of the objects that make up the system. The momentum of the system at the center of mass is the sum of the momenta of each object in the system, of which the center of mass is measured.
\begin{equation*}
	a_\text{cm}
	=\diff[2]{ x_\text{cm}}{t}
	=\frac{\sum m_ia_i}{\sum m_i}
	=\frac{\sum F}{M}
	\implies \boxed{\Sigma F=Ma_\text{cm}}
\end{equation*}
Is Newton's second law for the center of mass.

\theorem{Center of mass, velocity, acceleration, momentum and force}{
	\begin{align*}
		v_\text{cm}&=\diff{x_\text{cm}}{t}=\frac{\sum m_iv_i}{\sum m_i}=\frac{\sum p_i}{M} \\
		p_\text{cm}&=\sum m_iv_i=\sum p_i=Mv_ \text{cm} \\
		a_\text{cm}&=\diff{v_\text{cm}}{t}=\frac{\sum m_ia_i}{\sum m_i}=\frac{\sum F_i}{M} \\
		\Sigma F&=Ma_\text{cm}
	\end{align*}
	where $p_\text{cm}$ is the momentum at the center of mass, $p_i$ are individual momenta and $M=\sum m_i$ is the total mass of the system.

	The sum of the momenta of each object in the system ($\sum p_i$) is the  momentum of the system $p_\text{sys}$. Velocity of center of mass $v_\text{cm}=p_\text{sys}/M$.

	After an explosion, the system's center of mass continues along the original projectile's trajectory.

	If only internal forces act on a system, the center of mass does not move: $\Sigma F=Ma_\text{cm}=0 \implies a_\text{cm}=0$.
}

\theorem{Center of mass and internal forces}{
	If only internal forces act on a system, then the net external force $\Sigma F=0$, so the \ul{center of mass does not move.} This appears in problems involving two objects in a system, asking for the position of both objects after one moves relative to the other. The center of mass of the system does not change, so we can solve for the positions by setting the center of mass before equal to the center of mass after the movement.
}

\remark{Center of mass, rotation and translation}{
	If a force is applied on a free object, it will only translate if the force is in line with the center of mass, and will rotate and translate otherwise, because there is no net torque. The lever arm length is measured from where the force is applied to the pivot, in this case, the center of mass.
}

\subsection{Density}

Volume density is defined as mass per volume: $\rho=m/V$ with units \si{\kilo\gram\per\meter^3}.

In physics we use length density. Length density is mass per unit length: $\lambda=m/L$ where $L$ is length, and measured in \si{\kilo\gram\per\meter}.

Consider a solid that is not of uniform length density. These objects do not have its center of mass at their center.

Recall our original definition of center of mass for a system of particles, and change the discrete sum into a continuous integral:
\begin{equation*}
    x_ \text{cm}=\frac{\sum m_ix_i}{\sum m_i}
	=\frac{\int x\,\mathrm{d}m}{\int \,\mathrm{d}m}
\end{equation*}
where $\mathrm{d}m$ is the differential of mass (an infinitesimally small mass), by dividing the object into an infinite number of particles, each with mass $\mathrm{d}m$.

\theorem{Generalized center of mass}{
	The center of mass of a system with total mass $M$ is:
	\begin{equation*}
		x_\text{cm}=\frac{\int x\,\mathrm{d}m}{\int \,\mathrm{d}m}=\frac1M\int x\,\mathrm{d}m=\frac1M \int x\cdot\lambda\,\mathrm{d}x
	\end{equation*}
	where $\lambda$ is the linear mass density:
	\begin{equation*}
		\lambda = \diff mx
		\implies \mathrm{d}m=\lambda \,\mathrm{d}x
		\implies m=\int \lambda\,\mathrm{d}x
	\end{equation*}
}

\example{Rod with uniform linear density}{
	See Figure 4.1 for diagram. The mass of the rod is $M$ and the length of the rod is $L$. The linear mass density of the rod is $\lambda=M/L$.

	We divide the rod into small slices, each of mass $\mathrm{d}m$ and height/thickness $\mathrm{d}x$. Since the rod is of uniform linear density:
	\begin{equation*}
	    \lambda=\frac{M}{L}=\frac{\mathrm{d}m}{\mathrm{d}x}
		\implies \boxed{\mathrm{d}m=\lambda \,\mathrm{d}x}
	\end{equation*}
	The mass of the rod is the integral from the left of the rod to the right (bounds $0$ and $L$):
	\begin{equation*}
	    M=\int \,\mathrm{d}m=\int \lambda\,\mathrm{d}x=\lambda \int_{0}^{L}\,\mathrm{d}x
		=\lambda \left[x\right]_{0}^{L}
		=L\lambda
		=L \left( \frac ML \right)
		=M
	\end{equation*}
	So the denominator $\int \,\mathrm{d}m$ does equal the total mass $M$. Now we can determine the center of mass:
	\begin{equation*}
	    x_ \text{cm}
		=\frac{\int x\,\mathrm{d}m}{\int \,\mathrm{d}m}
		=\frac1M \int_{0}^{L}x\cdot\lambda\,\mathrm{d}x
		=\frac{\lambda}{M}\int_{0}^{L}x\,\mathrm{d}x
		=\frac{\lambda}{M} \left[\frac{x^2}{2}\right]_{0}^{L}
		=\frac ML \frac 1M \frac{L^2}{2}
		=\frac{L}{2}
	\end{equation*}
}

\example{Rod with non-uniform linear density}{
	Consider a rod with linear density $\lambda=ax+b$. The linear density of the rod varies by position, and is not uniform. The units for constants $a,b$ are \si{\kilo\gram\per\meter^2}, the unit for $\lambda$ is \si{\kilo\gram\per\meter} and the unit for $x$ is \si{\meter}. This is correct after canceling out all units with dimensional analysis.

	We can determine the total mass of the rod:
	\begin{equation*}
	    M
		=\int \,\mathrm{d}m
		=\int \lambda\,\mathrm{d}x
		=\int_{0}^{L}(ax+b)\,\mathrm{d}x
		=\left[\frac{ax^2}{2}+bx\right]_{0}^{L}
		=\frac{aL^2}{2}+bL
	\end{equation*}
	The center of mass is at:
	\begin{align*}
	    x_ \text{cm}
		&=\frac1M \int x\,\mathrm{d}m
		=\frac1M \int_{0}^{L}x\cdot\lambda\,\mathrm{d}x
		=\frac1M \int_{0}^{L}x(ax+b)\,\mathrm{d}x
		=\frac1M \int_{0}^{L}\left( ax^2+bx \right)\mathrm{d}x \\
		&=\frac1M \left[\frac{ax^3}{3}+\frac{bx^2}{2}\right]_{0}^{L}
		=\frac1M \left( \frac{aL^3}{3}+\frac{bL^2}{2} \right)
		=\frac{aL^3}{3M}+\frac{bL^2}{2M}
	\end{align*}

	For a rod of length $0.3\,\si{\meter}$ and with density given by $\lambda=ax+b$ where $a=6,b=10$:
	\begin{align*}
		M&=\frac{6(0.3)^2}{2}+10(0.3)=3.27\,\si{\kilo\gram} \\
		x_\text{cm}&=\frac{6(0.3)^3}{3(3.27)}+\frac{10(0.3)^2}{2(3.27)}=0.154\,\si{\meter}
	\end{align*}
}

\section{Rotation}
\label{sec:rotation}

\subsection{Circular Motion}

In uniform circular motion (UCM), a particle moves at constant speed (but not constant velocity, since direction changes). Centripetal acceleration $a_\text{c}$ and centripetal force $F_\text{c}$ are:
\begin{equation*}
	a_\text{c}=\frac{v^2}{r}
	\implies F_\text{c}=\frac{mv^2}{r}
\end{equation*}
where $v$ is speed/magnitude of velocity, $r$ is the radius of rotation and $m$ is the particle's mass. Acceleration and force are normal to the circle and point towards the radius.

If an object is resting on a spinning disk, the centripetal force needed to keep the object in circular motion is $F_c=mv^2/r$. The maximum static friction on the object $F_s=\mu mg$. If:
\begin{equation*}
    F_c>F_s
	\implies \frac{mv^2}{r}>\mu mg
	\implies a_c=\frac{v^2}{r}>\mu g
\end{equation*}
Then the object will not move in a circular pattern. $a_c$ is centripetal acceleration and can be written a number of different ways by substituting values for velocity $v$.

Tangential acceleration is acceleration normal to the vector to the center of rotation and is the same as linear acceleration: $a_\text{tangential}=\Delta v/\Delta t= \mathrm{d}v/\mathrm{d}t$.

The magnitude of the total acceleration is:
\begin{equation*}
	a_ \text{total}=\sqrt{(a_\text{c})^2+(a_ \text{tangential})^2}
\end{equation*}

Weight $w=mg$ resolves into two vectors, one tangent to the circle and one normal to it.

\subsubsection{Banked Curve}

On a flat curve, the force pushing the vehicle towards the center of the curve and preventing it from sliding off the road is static friction, because the wheels of the car are stationary relative to the road.

In a banked curve, there is a component of the normal force towards the center of the curve, which helps keep the vehicle on the road. Figure 5.1 is a free body diagram of a car on a banked curve.

The net force in the $x$ direction (parallel to the center of the curve) equals $ma$ by Newton's second law. Acceleration is centripetal acceleration $a_ \text{c}=v^2/r$:
\begin{align*}
	\Sigma F_x&=ma
	\implies N_x=\frac{mv^2}{r}
	\implies N\sin\theta=\frac{mv^2}{r} \\
	\intertext{And in the $y$ direction, the net force is $0$ since the car remains on the ground:}
	\Sigma F_y&=0
	\implies N_y=w=mg
	\implies N\cos\theta=mg
\end{align*}
Dividing these two equations:
\begin{equation*}
    \tan\theta=\frac{mv^2}{rmg}=\frac{v^2}{rg}
	\implies v_ \text{critical}= \sqrt{rg\tan\theta}
\end{equation*}
Without friction (which includes wheel turning), $v_ \text{critical}$ is the exact speed such that the car will travel in circular motion around the curve without sliding.

If there is friction, the force of friction $F_ \text{f}$ is parallel to the banked curve. If $v<v_ \text{critical}$, the friction is up the banked curve, and down the banked curve if $v> v_ \text{critical}$. See Figure 5.2.

\subsection{Orbital Motion}

Gravitational force $F_g$ is equal to centripetal force $F_c$ on the orbiting object:
\begin{equation*}
    F_g
	=\frac{GMm}{r^2}
    =\frac{mv^2}{r}
	=F_c
\end{equation*}
Solving for velocity $v$ of the orbiting object from above, and by dividing circumference $2\pi r$ with period $T$:
\begin{equation*}
	v=\sqrt{\frac{GM}{r}}
	=\frac{2\pi r}{T}
	=2\pi r f
\end{equation*}
Multiplying these equations:
\begin{equation*}
    \frac{r^3}{T^2}
	=\frac{GM}{4\pi^2}
\end{equation*}

\theorem{Kepler's third law}{
	If an object is orbiting a larger of mass $M$, the period $T$ and radius of its orbit $r$ are related by, where $G$ is the gravitational constant:
	\begin{equation*}
		\frac{r^3}{T^2}=\frac{GM}{4\pi^2}
	\end{equation*}
	$T^2$ is proportional to $r^3$.
}

\subsection{Rotational Kinematics}

\theorem{Rotational kinematics basics}{$\theta$ is angular position, $\omega$ is angular velocity, and $\alpha$ is angular acceleration. Linear kinematics equations can be applied; $\theta$ is $x$, $\omega$ is $v$, and $\alpha$ is $a$.
\begin{alignat*}{2}
	v
	&=\frac{\Delta x}{\Delta t}=\diff xt
	&\qquad\qquad
	\omega
	&=\frac{\Delta \theta}{\Delta t}=\diff \theta t \\
	a
	&=\frac{\Delta v}{\Delta t}=\diff vt
	&\qquad\qquad
	\alpha
	&=\frac{\Delta \omega}{\Delta t}=\diff \omega t
\end{alignat*}
Where acceleration $a,\alpha$ is constant:
\begin{alignat*}{2}
	v 
	&=v_0+at
	&\qquad\qquad
	\omega 
	&=\omega_0+\alpha t \\
	v^2
	&=v_0^2+2a\Delta x
	&\qquad\qquad
	\omega^2
	&=\omega_0^2+2\alpha \Delta \theta \\
	\Delta x
	&=v_0t+\frac12at^2
	&\qquad\qquad
	\Delta \theta
	&=\omega_0t +\frac12\alpha t^2
\end{alignat*}
}

As before, change in angular position $\Delta \theta=\theta_ \text{f}- \theta_ \text{i}$. If $\theta$ is measured in radians, then the arc length $s$ is $r\theta$, by definition of radians. The change in arc length $\Delta s=r\Delta \theta$, since radius $r$ does not change. Radius $r$ is the distance from the center of rotation to the point of interest.
\begin{equation*}
    \Delta s=r\Delta \theta=r\Delta x
	\implies \Delta x=\Delta \theta=\frac{\Delta s}{r}
\end{equation*}
Recall from previously that $\omega=\Delta \theta/\Delta t$. Change in arc length $\Delta s$ is change in linear position $\Delta x$.
\begin{equation*}
    \omega
	=\frac{\Delta \theta}{\Delta t}
	=\frac{\Delta s/r}{\Delta t}
	=\frac 1r \frac{\Delta s}{\Delta t}
	=\frac{v}{r}
	\implies v=\omega r
\end{equation*}
Similarly, for acceleration:
\begin{equation*}
    a=\alpha r
	\implies 
	\alpha=\frac{a}{r}
\end{equation*}
\theorem{Connection between rotational and linear quantities}{
	\begin{equation*}
		\Delta x=\Delta s=r\Delta \theta
		\qquad
		v=r \omega 
		\qquad
		a=r \alpha
	\end{equation*}
}
In circular motion, centripetal acceleration $a_ \text{c}$ is:
\begin{equation*}
    a_ \text{c}
	=\frac{v^2}{r}
	=\frac{(r\omega)^2}{r}
	=\frac{r^2\omega^2}{r}
	=r\omega^2
\end{equation*}
Recall from before, that the magnitude of total acceleration of a particle in circular motion is:
\begin{align*}
	a_ \text{total}
	&=\sqrt{(a_\text{c})^2+(a_ \text{tangential})^2} \\
	\intertext{Tangential acceleration $a$ can be written in terms of angular acceleration: $a=r\alpha$.}
	&=\sqrt{\left( r\omega^2 \right)^2+(r\alpha)^2}
	=r \sqrt{\omega^4+\alpha^2t}
\end{align*}

\section{Rotational Inertia}

\definition{\textbf{Inertia} is a property of a body or object that defines its resistance to change.}

Inertia is related to mass; the greater the mass, the greater the inertia. Newton's second for linear motion is:
\begin{equation*}
    \Sigma F=ma
\end{equation*}
\theorem{Newton's second law for rotation}{
	\begin{equation*}
	    \Sigma \tau=I \alpha
	\end{equation*}
	where $\tau$ is torque, $\alpha$ is angular acceleration and $I$ is the moment of inertia/rotational inertia.
}

The moment of inertia $I$ is a property of the body that defines its resistance to change in angular velocity about an axis of motion. It takes mass, distribution of mass and position of rotation axis into account.

Suppose we have a point mass of mass $m$, $r$ units away from the axis of rotation. Starting with Newton's second law, multiply by $r$, and recall that linear acceleration $a=\alpha r$:
\begin{equation*}
    r\Sigma F=mar
	\implies
	\Sigma \tau=m(\alpha r)r
	=\left( mr^2 \right)\alpha
\end{equation*}
So the moment of inertia $I$ for this point mass is $mr^2$.

\theorem{Moment of inertia}{
	The moment of inertia of a point mass with mass $m$ and distance $r$ from the axis of rotation is:
	\begin{equation*}
		I=mr^2
	\end{equation*}
	Moment of inertia $I_ \text{total}$ of a system is the sum of the moments of inertia of the objects within the system:
	\begin{align*}
		I_ \text{total}&=\sum I_i \\
		\intertext{For a system composed of point masses:}
		I_ \text{total}&=\sum m_ir_i^2
	\end{align*}
	where $m_i,r_i$ are the mass and radius of each point mass.

	Moment of inertia as an integral is:
	\begin{equation*}
	    I
		=\int r^2\,\mathrm{d}m
	\end{equation*}
	where $r$ is distance to the axis of rotation. The linear mass density function $\lambda$ may also be given: $\lambda=\mathrm{d}m/\mathrm{d}r \implies \mathrm{d}m=\lambda \,\mathrm{d}r$:
	\begin{equation*}
	    I=\int r^2\lambda \,\mathrm{d}r
	\end{equation*}
}

\begin{equation*}
    \Sigma \tau = I\alpha
	\implies I=\frac{\Sigma \tau}{\alpha}
\end{equation*}
The units of torque is newton meter (\si{\newton\meter}). The units of angular acceleration is radian per second square (\si{\radian\per\second^2}). So the units for moment of inertia $I$ is:
\begin{equation*}
	\left[\frac{\si{\newton\meter}}{\si{\radian\per\second^2}}\right]
	=\left[\frac{\si{(\kilo\gram\cdot\meter\per\second^2)\cdot\meter\cdot\second^2}}{\si{\radian}}\right]
	=\left[\si{\kilo\gram\cdot\meter^2}\right]
\end{equation*}

\theorem{Rotation about the center of mass}{
	In Figure 6.3, The moment of inertia of rotation about the center of mass is:
	\begin{equation*}
		I_\text{cm}=\frac12mL^2
	\end{equation*}
}

\theorem{Parallel axis theorem}{
	\begin{equation*}
		I_\text{new}=I'=I_\text{cm}+Mx^2
	\end{equation*}
	where $I_\text{cm}$ is the moment of inertia for a rotation axis through the center of mass, $M$ is the total mass of the system and $x$ is the distance from the center of mass to the new axis of rotation.
}

\subsection{Extended Objects}

\theorem{Moments of inertia of certain extended objects}{
A cylinder with height $H$, radius $R$ and mass $M$ is rotating about its central axis.
\begin{align*}
    I&=\frac12MR^2 \\
	\intertext{A rod/stick has length $L$ has negligible thickness and is rotating horizontally about its center.}
    I&=\frac{1}{12}ML^2 \\
	\intertext{A solid sphere of radius $R$ and mass $M$ is rotating about its central axis.}
    I&=\frac25MR^2 \\
	\intertext{A hollow sphere of radius $R$ and mass $M$ is rotating about its central axis.}
	I&=\frac23MR^2
	\intertext{A ring/hoop is a hollow cylinder without bases, and has height $H$, one radius $R$ (the thickness of the surface is negligible) and mass $M$.}
	I&=MR^2
\end{align*}
}

\subsection{Moment of Inertia of a Uniform Rod}

See Figure 6.4.

We have linear mass density $\lambda=\mathrm{d}m/\mathrm{d}x\implies \mathrm{d}m=\lambda \,\mathrm{d}x$. Since the rod is uniform, $\mathrm{d}m=(M/L)\,\mathrm{d}x$.
\begin{equation*}
	\int r^2\,\mathrm{d}m
	=\int r^2 \left( \frac{M}{L} \right)\mathrm{d}x
	=\int x^2 \left( \frac{M}{L} \right)\mathrm{d}x
\end{equation*}
where $r$ is distance from axis of rotation, so $r$ is $x$.

The length of the rod is $L$, so the bounds of integration are $-L/2$ and $L/2$.
\begin{equation*}
	\int_{-L/2}^{L/2}x^2 \left( \frac ML \right)\mathrm{d}x
	=\frac ML \left[\frac{x^3}{3}\right]_{-L/2}^{L/2}
	=\frac13 \frac ML \left( \frac{L^3}{8}- \left( -\frac{L^3}{8} \right) \right)
	=\frac13 \frac ML \frac{L^3}{4}
	=\boxed{\frac{1}{12}ML^2}
\end{equation*}

Application of parallel axis theorem are also in the notebook.

\subsection{Moment of Inertia of a Non-Uniform Rod}

Suppose we have linear mass density $\lambda$ as a function of $x$. As before, we have $\mathrm{d}m=\lambda \,\mathrm{d}x$. Distance $r$ is equal to $x$. Then the moment of inertia is given by:
\begin{equation*}
    I
	=\int_a^b r^2\,\mathrm{d}m
	=\int_a^b \lambda x^2 \,\mathrm{d}x
\end{equation*}

\subsection{Rotational Kinetic Energy}

\theorem{Rotational kinetic energy}{
	\begin{equation*}
	    K_ \text{rot}=\frac12 I\omega^2
	\end{equation*}
	where $I$ is moment of inertia and $\omega$ is angular velocity.
}
Checking units:
\begin{align*}
	K_ \text{rot}&=\left[ \si{\joule} \right]=\left[ \si{N\cdot m} \right]= \left[ \si{\kilo\gram\cdot \frac{\meter}{\second^2}\cdot \meter} \right]=\left[ \si{\kilo\gram\cdot \frac{\meter^2}{\second^2}} \right] \\
	I&=\left[ \si{\kilo\gram\cdot\meter^2} \right] \\
	\omega=\frac{v}{r}
	&=\left[ \si{\frac{\meter\per\second}{\meter}} \right]
	=\left[ \si{\frac{1}{\second}} \right] \\
	\frac12 I\omega^2
	&= \left[ \si{\kilo\gram\cdot\meter^2\cdot \frac{1}{\second^2}} \right]
	= \left[ \si{\kilo\gram\cdot \frac{\meter^2}{\second^2}} \right]
\end{align*}

\theorem{Rotational work}{
	For constant torque:
	\begin{equation*}
	    w=\tau\cdot\Delta\theta
	\end{equation*}
}

\example{Energy and kinematics}{
	A pulley of radius $R=0.15\,\si{\meter}$ and mass $M=8\,\si{\kilo\gram}$ is connected by a string to a block of mass $m=5\si{\kilo\gram}$.

	Using kinematics: let $a$ be the acceleration of the block and the linear acceleration of the pulley, which are the same. The weight is $w=mg=50$ and $T$ is tension in the string. Force accelerating the block is $w-T$:
	\begin{equation*}
	    w-T=ma \implies T=w-ma=mg-ma
	\end{equation*}
	Tension accelerates the pulley. Torque on the pulley is $TR$. Linear acceleration $a$ (which equals acceleration of the block) is related to angular acceleration by $a=\alpha R$. The angular acceleration of the pulley is:
	\begin{equation*}
	    \tau=I\alpha
		\implies \tau=TR=\frac12MR^2\cdot \left( \frac{a}{R} \right)
		\implies TR=\frac12 MRa
		\implies T=\frac12 Ma
	\end{equation*}
	Combining these equations:
	\begin{equation*}
	    mg-ma=T=\frac12Ma
		\implies 50-5a=\frac12(8)a=4a
		\implies 9a=50
		\implies a=5.56
	\end{equation*}
	We can then use kinematics to solve for the final velocity.

	Using energy: gravitational potential energy $U_g$ only depends on the block, and equals the sum of kinetic energy of the block on impact and the rotational kinetic energy of the pulley on impact
	\begin{equation*}
		U_g=K_ \text{block}+K_ \text{rot,pulley}
		\implies mg\Delta h=\frac12mv^2+\frac12I \left( \frac vr \right)^2
	\end{equation*}
	Substituting the moment of inertia $I=MR^2/2$ and solving for $v$ gives the final velocity.
}

\subsection{Rolling}

Rolling is a combination of pure translation and pure rotation, and there is no velocity at the point of contact: Figure 6.5.

If the velocity at center of mass $v_ \text{cm}=\omega R$, the object is rolling without slipping. If $v_ \text{cm}\neq\omega R$, the object is rolling with slipping.

Rolling is a combination of pure rotation and translation. Connecting equations for rolling:
\begin{align*}
	v_ \text{cm}&=\omega R \tag{Condition for rolling} \\
	\Delta x_ \text{cm}&=\Delta \theta R \\
	a_ \text{cm}&=\alpha R
\end{align*}

\subsubsection{Kinetic Energy in Rolling}

\theorem{Rolling kinetic energy}{Kinetic energy of a rolling object is a combination of linear kinetic energy and rotational kinetic energy:
\begin{equation*}
    K_ \text{rolling}= K_ \text{linear}+ K_ \text{rot}=\frac12m v_ \text{cm}^2+\frac12I_ \text{cm}\omega^2
\end{equation*}
This quantity can be equated to gravitational potential energy $U_g=mg\Delta h$ to find kinetic energy at the bottom of an inclined slope.}

\example{Rolling disk}{
Consider an object rolling on an inclined plane (Figure 6.6). We want to calculate the final velocity of the object at the bottom of the plane. The initial kinetic energy equals the final kinetic energy due to rolling:
\begin{align*}
    U_ \text{g,i}=K_ \text{rolling,f}
	&\implies \boxed{MgH=\frac12M v_ \text{cm}^2+\frac12 I_ \text{cm}\omega^2}
	\intertext{Since we are working with a disk, the moment of inertia is $I_ \text{cm}=MR^2/2$, and angular velocity $\omega=v_ \text{cm}/R$.}
	&\implies MgH=\frac12Mv_ \text{cm}^2+\frac12 \left( \frac12MR^2 \right)\left( \frac{v_ \text{cm}}{R} \right)^2 \\
	&\implies gH=\frac12v_ \text{cm}^2+\frac14v_ \text{cm}^2 \cancel{\frac{R^2}{R^2}} \\
	&\implies gH=\frac34v_ \text{cm}^2
	\implies \boxed{v_ \text{cm}=\sqrt{\frac{4}{3}gH}}
\end{align*}
}
\example{Moment of inertia of rolling object}{In general, for an object of moment of inertia $I=x MR^2$:
\begin{align*}
    mgH=
    U_ \text{g,i}=K_ \text{rolling,f}
	&\implies MgH=\frac12M v_ \text{cm}^2+\frac12 I_ \text{cm}\omega^2 \\
	&\implies MgH=\frac12M v_ \text{cm}^2+\frac12 \left( xMR^2 \right) \left( \frac{v_\text{cm}}{R} \right)^2 \\
	&\implies MgH=\frac12M v_ \text{cm}^2+\frac12 xMv_\text{cm}^2\cancel{\frac{R^2}{R^2}} \\
	&\implies gH=\frac12 v_ \text{cm}^2+\frac12 xv_\text{cm}^2 \\
	&\implies gH=\left( \frac12+\frac12x \right) v_ \text{cm}^2 xv_\text{cm}^2 \\
	&\implies gH=\left( \frac12+\frac12x \right) v_ \text{cm}^2 \\
	&\implies v_ \text{cm}=\sqrt{\frac{gH}{\frac12+\frac12x}}
\end{align*}
For a solid sphere, $I_ \text{cm}=2MR^2/5$:
\begin{align*}
	v_ \text{cm}
	&=\sqrt{\frac{gH}{\frac12+\frac12\frac25}}
	=\sqrt{\frac{gH}{\frac{5}{10}+\frac{2}{10}}}
	=\sqrt{\frac{10}{7}gH}
	\intertext{For a hoop, $I_ \text{cm}=MR^2$:}
	v_ \text{cm}
	&=\sqrt{\frac{gH}{\frac12+\frac12(1)}}
	=\sqrt{\frac{gH}{1}}
	=\sqrt{gH}
	\intertext{For a hollow sphere, $I_ \text{cm}=2MR^2/3$:}
	v_ \text{cm}
	&=\sqrt{\frac{gH}{\frac12+\frac12\frac23}}
	=\sqrt{\frac{gH}{\frac36+\frac26}}
	=\sqrt{\frac65gH}
\end{align*}
}
The final velocity only depends on the shape of the object. \ul{The greater the moment of inertia, the slower translationally the object will be.} The more the mass is away from the center of rotation, the less translational kinetic energy there will be at the bottom of the slope. A larger portion of energy is used to rotate the object, because $I$ is greater (recall $\tau=I\alpha\implies \alpha=\tau/I$ is inversely proportional to $I$).

\subsubsection{Friction and Torque}

Consider the forces acting on disk ($I=MR^2/2$ rolling down an inclined sphere (Figure 6.7). As shown, net torque causing rotation is from static friction $F_s$:
\begin{equation*}
    \Sigma \tau=I\omega
	\implies F_sR=\left( \frac12MR^2 \right)\left( \frac{a}{R} \right)
	\implies F_sR=\frac12MRa
	\implies F_s=\frac12Ma
\end{equation*}
For translation, net force in the $x$ direction (parallel to the slope):
\begin{equation*}
    \Sigma F_x=ma
	\implies mg\sin\theta-F_s=ma
\end{equation*}
Combining these two equations:
\begin{equation*}
    mg\sin\theta-\frac12Ma=ma
	\implies \sin\theta=\frac32a
	\implies a=\frac23g\sin\theta
\end{equation*}
Similar procedure for other objects with different moments of inertia $I$.
\theorem{Rolling linear acceleration}{
	Static friction exerts a force distance $R$ from the center of the rolling object, producing a torque of $F_sR$. Acceleration down the slope depends on the horizontal net force $\Sigma F_x$, which is the horizontal component of gravity $mg\sin\theta$ minus static friction $F_s$.
}
If there is no friction, the object \ul{will not roll}, and will only slide down the slope.

\subsubsection{Rolling with Slipping}

Condition for pure rolling ($\omega$ is angular velocity, $R$ is radius):
\begin{equation*}
	v_\text{cm}=\underbrace{\omega R}_\text{tangential velocity}
\end{equation*}
When an object is rolling down an inclined plane, static friction $F_\text{s}$ causes the rotational part of rollling. $F_\text{s}$ does no work so energy is not dissipated. The object will continue rolling even if there is no more friction.

Properties of rolling with slipping:
\begin{itemize}
	\item When an object rolls and skips, there is kinetic friction $F_\text{k}$ involved.
	\item $F_\text{k}$ affects both $v_\text{cm}$ and $\omega$.
	\item Rollingg with slipping is \textit{transient} (unstable, changes over time) because $F_\text{k}$ will eventually cause $v_\text{cm}$ and $\omega R$ to become equal, at which point the object rolls without slipping.
	\item Energy \ul{is} dissipated (mechanical energy transferred to heat/thermal energy).
\end{itemize}

\ul{Bowling ball where the first part of the lane is smooth/have zero friction.} In the first part, the ball only translates: at $t=0,\omega_0=0\implies v_\text{cm}>\omega_0 R$. In the part with friction, there is slipping governed by $\Sigma F_x=ma$:
\begin{equation*}
    \Sigma F_x=ma
	\implies -F_\text{k}=-\mu_\text{k}N=-\mu_\text{k}mg=ma
	\implies a=-u_kg
\end{equation*}
This value of acceleration can be used to calculate $v_\text{cm}$. As time increases, $v_\text{cm}$ decreases. Rolling is governed bt $\Sigma \tau=I\alpha$. Torque is $F_\text{k}R$ (positive because $F_\text{k}R$ causes the rotation).
\begin{equation*}
    \Sigma \tau=ma
	\implies F_\text{k}R=\mu_\text{k}NR=\mu_\text{k}mgR=I\alpha
	\implies \alpha=\frac{\mu_\text{k}mgR}{I}
\end{equation*}
Angular acceleration $\alpha$ can be used to calculate angular velocity. As time increases, $\omega$ and therefore $\omega R$ increases. At some point in this area with friction, $v_\text{cm}=\omega R$ and rolls without slipping.

\ul{Backspin.} The object spins backwards and moves forward, but will change direction and translate backwards while rolling without slipping. As before, $a=-\mu_\text{k}g$ so $v_\text{cm}$ decreases as time increases. In this case, torque is negative; it decreases angular velocity instead of increasing it.
\begin{equation*}
	\alpha=-\frac{\mu_\text{k}mgR}{I}
\end{equation*}
As time increases, $\omega$ and $\omega R$ decreases. At some point, $v_\text{cm}$ slows down to zero and turns around. It will roll backwards without slipping. This turnaround point occurs when $v_\text{cm}=0 \iff v_\text{cm}=v_0+at=v_0-\mu_\text{k}gt=0$. Angular velocity $\omega$ at this turn around point can be calculated with the $t$ value calculated before and $\alpha$.

\ul{Topspin.} $F_\text{k}$ increases $v_\text{cm}$:
\begin{equation*}
	a=\mu_\text{k} g
	\implies v_\text{cm}=v_0+\mu_\text{k}gt
\end{equation*}

\subsection{Determining Moment of Inertia}

In Figure 6.8, we have a rotational and a translational part:
\begin{alignat*}{2}
	\Delta \tau&=I\alpha
	&\qquad
	\Delta F_y&=ma \\
	Tr&=I\alpha
	&\qquad
	mg-T&=ma \\
	&&\qquad
	T&=mg-ma
\end{alignat*}
\begin{equation*}
    Tr=I\alpha
	\implies (mg-ma)r=I \left( \frac{a}{r} \right)
	\implies \boxed{I=\frac{mr^2(g-a)}{a}}
\end{equation*}
And $a$ can be calculated by recording the time for the mass to fall to the ground, with a known $\Delta y$, which can be found by the kinematics equation:
\begin{equation*}
	\Delta y
	=\cancelto{0}{v_0t}+\frac12at^2
	=\frac12at^2
	\implies 
	a=\frac{2\Delta y}{t^2}
\end{equation*}
Experimental errors:
\begin{itemize}
	\item Unwinding of string: Radius $R$ is not constant (string is ``on top'' of the wound string), and there is friction associated with unwinding.
	\item String may not leave perpendicular, or $\theta\neq90^\circ$ and $\Sigma\tau\neq TR$
	\item String may not be completely horizontal, or only a component of tension is providing torque.
	\item Mass is swinging.
\end{itemize}
Improvements:
\begin{itemize}
	\item Make sure eye level with mass and meter stick to calculate $\Delta y$ more accurately.
	\item Use automated method of timing.
	\item Increase $\Delta y$: reduces the error involved with starting and stopping the stopwatch (calculation of $\Delta t$), since error is smaller relative to $\Delta t$.
	\item \ul{Increase the number of trials with constant experimental conditions.}
	\item Increase the number of experimental conditions (by using more masses).
\end{itemize}

\subsection{Angular Momentum}

\remark{Review of linear momentum}{
	\begin{equation*}
	    p=mv
	\end{equation*}
	Units of linear momentum are \si{\kilo\gram\cdot\meter\per\second} or \si{\newton\cdot\second}. Momentum is related to impulse $J$:
	\begin{equation*}
	    J=\Sigma F\cdot\Delta t=\Delta p=m\Delta v=\int F\,\mathrm{d}t
		\implies \Delta F=\diff pt
	\end{equation*}
	When there are multiple objects, if there is no external force acting on the system, momentum is conserved:
	\begin{equation*}
	    \Delta p_ \text{system}=0
		\iff
		\sum p_ \text{initial}=\sum p_ \text{final}
	\end{equation*}
	In AP Physics C we assume $\Delta F_ \text{ext}=0$ and momentum is conserved.
}

From the equation of linear momentum $p=mv$, the rotational equivalent of mass is moment of inertia $I$ and the equivalent of linear velocity $v$ is angular velocity $\omega$. Therefore angular momentum $L=I\omega$.

Alternatively, recall the impulse-change in momentum equation:
\begin{equation*}
    J=\Delta p=\Sigma F\cdot \Delta t=\Delta(mv)
\end{equation*}
Multiplying by $r\sin\theta$, the lever arm length, on both sides:
\begin{equation*}
	\underbrace{(r\sin\theta)\Sigma F}_{\Sigma \tau}\cdot \Delta t
	=\underbrace{(r\sin\theta)\Delta (mv)}_{\text{change in $L$}}
	\implies \Sigma \tau \cdot \Delta t=\Delta (mvr\sin\theta)
\end{equation*}
Another way of thinking of this is realizing that like how linear momentum $p=mv$ is the derivative with respect to time of force $F=ma$, angular momentum is the derivative of torque with respect to time.
\begin{equation*}
	\vec L
	=\diff{\tau}{t}
	=\deriv[t](r\times F)
	=r\times \diff Ft
	=r\times p
	=r\times (mv)
	\implies
	L=rmv\sin\theta
\end{equation*}

Conservation of angular momentum:
\begin{align*}
	m_1v_1r_1\sin(\theta_1)
	&=m_2v_2r_2\sin(\theta_2)
	\intertext{If $\theta=\theta_1=\theta_2=90^\circ$, then $\sin\theta=1$:}
	m_1v_1r_1
	&=m_2v_2r_2
	\intertext{And if mass does not change; $m_1=m_2$:}
	v_1r_1
	&=v_2r_2
\end{align*}
By this principle, in orbital mechanics, objects are faster if their orbit is lower (see Kepler's laws).

\theorem{Angular momentum}{
	\begin{equation*}
	    L=I\omega=\lVert r\times p \rVert=\lVert r\times (mv) \rVert =mvr\sin\theta
	\end{equation*}
	where $L$ is angular momentum, $I$ is moment of inertia and $\omega$ is angular velocity. Angular momentum has units of \si{\kilo\gram\cdot\meter^2\per\second}. Angular momentum is conserved (conservation of linear momentum $\sum p_i=\sum p_f$):
	\begin{equation*}
		\sum L_\text{i}=\sum L_\text{f}
		\iff \sum I_\text{i}\omega_\text{i}=\sum I_\text{f}\omega_\text{f}
	\end{equation*}
}

So we can still calculate angular momentum for an object moving in a straight line for a chosen reference point.

Dimensional analysis on $L=I\omega=mvr\sin\theta$ yields equal units:
\begin{equation*}
    I\omega
	=\left[ \si{\kilo\gram\cdot\meter^2}\cdot \si{1/\second} \right]
	=\left[ \si{\kilo\gram\cdot\meter^2\per\second} \right]
	=\Bigl[ \si{\kilo\gram\cdot\meter\per\second\cdot\meter} \Bigr]
	=mvr\sin\theta
\end{equation*}

\theorem{Angular impulse}{
	\begin{equation*}
	    \Delta L 
		=\int_{t_0}^{t_1}\tau\,\mathrm{d}t \\
		=\underbrace{\Sigma \tau \cdot \Delta t
		=\Sigma \tau \cdot (t_1-t_0)}_{\text{constant torque}}
	\end{equation*}
	This is similar to the definition of change in linear momentum $\Delta p=\int_{t_0}^{t_1}F\,\mathrm{d}t$.
}

\subsubsection{Conservation of Momentum}

In general, linear momentum and angular momentum are \ul{both conserved.} Kinetic energy is \ul{not} always conserved.

If one point of the stick is anchored at one point, linear momentum is not conserved because the anchor exerts a net external force on the ball-stick system. Angular momentum is conserved because the anchor exerts no torque (lever arm length is $0$).

\subsection{Kepler's Laws}

\theorem{Kepler's laws}{
	\begin{enumerate}
		\item Planetary orbits are elliptical with the sun at a focus point.
		\item An imaginary line between the planet and the sun ($r$) sweeps equal areas in equal time intervals.
		\item For a given object, $T^2$ is directly proportional to $r^3$, where $T$ is the orbit's period. Furthermore:
		\begin{equation*}
			\frac{T^2}{r^3}=\frac{4\pi^2}{GM}
		\end{equation*}
		where $G$ is the universal gravitational constant, and $M$ is the mass of the object being orbited.
	\end{enumerate}
}

\section{Simple Harmonic Motion}

\definition{
	\textbf{Periodic motion} is motion that repeats over a fixed period of time.

	\textbf{Oscillatory motion} is periodic motion where an object moves back and forth about a mean/equilibrium.

	\textbf{Simple harmonic motion} is oscillatory motion where there is a restoring force that is \ul{directly proportional} to the displacement of the oscillator.
}

In a spring, the restoring storing force is $k\Delta x$, which is directly proportional to displacement.

In a pendulum, the restoring force is $mg\sin\theta$. It is not directly proportional to displacement, so it is not simple harmonic motion. (For small angles, use the small angle approximation $\sin\theta\approx\tan\theta\approx\theta$.)

\subsection{Sine and Cosine Equation}

Position, velocity, acceleration and force can be written as sine and cosine equations. The position $x$ can be generally written as:
\begin{equation*}
    x=x_ \text{max} \cos\left(\omega t+\varphi_0\right)
\end{equation*}
where $\varphi_0$ is the phase shift, $x_ \text{max}$ is the amplitude or radius (maximum displacement, since cosine is bound between $-1$ and $1$), and $\omega$ is angular velocity: $\omega=\Delta\theta/\Delta t=2\pi/T=2\pi f$.

To properly describe uniform circular motion, we need an $x$ as a cosine equation and a $y$ as a sine equation.

\theorem{Sine and cosine equation}{
The general sine equation and the general cosine equation are:
\begin{align*}
    y&=A\sin(\omega t+\varphi_0) \\
    y&=A\cos(\omega t+\varphi_0)
\end{align*}
where:
\begin{align*}
	y&=\text{position of oscillator on the $y$-axis (m)} \\
	A&=\text{amplitude/radius/$y_\text{max}$ (m)} \\
	\omega&=\text{angular velocity or angular frequency (rad/s or 1/s)} \\
	\varphi_0&=\text{initial phase (angle) (rad)}
\end{align*}
At time $t=0$, $y=A\sin(\omega (0)+\varphi_0)=A\sin\varphi_0$ from which the phase shift $\varphi_0$ can be found.

The period (time between troughs/peaks) is $2\pi/\omega$.
}


\subsection{Springs In Series and In Parallel}

A spring is like a string in that the tension is constant throughout the spring. If we divide a spring into smaller springs, they will have the \ul{same restoring force}, which means the smaller springs will have a higher spring constant. Figure 7.1.

In series: Suppose we have two springs attached to each other, the lower attached to a mass $m$ with forces $F_1,F_2$ and spring constant $k_1,k_2$. We have another single spring attached to the same mass $m$ with forces $F_\text{equivalent}$ and spring constant $k_\text{equivalent}$. As before, the forces are all equal: $F_1+F_2=F_\text{equivalent}$. The displacement is also the same: $\Delta x_1+\Delta x_2=\Delta x_\text{equivalent}$. From Hooke's law $F=k\Delta x\implies x=F/k$, and since all the forces are equal:
\begin{equation*}
	\frac{F_1}{k_1}+\frac{F_2}{k_2}=\frac{F_\text{equivalent}}{k_\text{equivalent}}
	\implies \boxed{\frac{1}{k_1}+\frac{1}{k_2}=\frac{1}{k_\text{equivalent}}}
\end{equation*}

In parallel: two separate springs are both attached to a mass $m$. Using the same notation, and $k=F/\Delta x$:
\begin{equation*}
	\left\{\begin{gathered}
		\Delta x=\Delta x_1=\Delta x_2=\Delta x_\text{equivalent} \\
		F_1+F_2=F_\text{equivalent}
	\end{gathered}\right\}
	\implies \frac{F_1}{\Delta x}+\frac{F_2}{\Delta x}=\frac{F_\text{equivalent}}{\Delta x}
	\implies \boxed{k_1+k_2=k_\text{equivalent}}
\end{equation*}
Springs are in parallel if:
\begin{itemize}
	\item All springs experience the same $\Delta x$.
	\item The total force applied to the mass is distributed among the individual springs.
	\item Each spring has a direct connection to the mass.
\end{itemize}

\subsection{Simple Harmonic Oscillator}

Suppose the position is defined as, where $A$ is amplitude, or the maximum value of $y$:
\begin{align*}
	y&=A \sin\left(\omega t+\varphi_0\right) \\
	v=\diff yt&=A\omega \cos\left(\omega t+\varphi_0\right) \\
	a=\diff vt&=-\omega^2 \underbrace{A\sin\left(\omega t+\varphi_0\right)}_{y}
	=-\omega^2 y
\end{align*}
\theorem{Simple harmonic motion position and acceleration}{
	The position of the simple harmonic oscillator can be expressed as the differential equation:
	\begin{equation*}
		\diff[2]{y}{t}=-\omega^2y
		\qquad
		\diff[2]{x}{t}=-\omega^2x
		\qquad
		\diff[2]{\theta}{t}=-\omega^2\theta
	\end{equation*}
}
\theorem{Period of various systems}{
	\begin{alignat*}{2}
		\text{Spring-mass system:}& \qquad T&&=2\pi \sqrt{\frac mk} \\
		\text{Simple pendulum:}& \qquad T&&=2\pi \sqrt{\frac lg} \\
		\text{Physical pendulum:}& \qquad T&&=2\pi \sqrt{\frac{I_\text{rot}}{rmg}} \\
		\text{Torsional pendulum:}& \qquad T&&=2\pi \sqrt{\frac {I_\text{obj}}k}
	\end{alignat*}
	where:
	\begin{align*}
		T&=\text{period} \\
		m&=\text{mass} \\
		k&=\text{spring/torsional constant} \\
		l&=\text{length of the string (simple pendulum)} \\
		I_\text{rot}&=\text{moment of inertia about the axis of rotation (physical pendulum)} \\
		r&=\text{distance from axis of rotation to center of mass (physical pendulum)} \\
		I_\text{obj}&=\text{moment of inertia of the object (torsional pendulum)}
	\end{align*}
}

\subsubsection{Period of Spring-Mass System}

There is a mass $M$ oscillating with amplitude $A$. Figure 7.3.1.
\begin{align*}
    \Sigma F
	&=ma \\
	-kx
	&=m\cdot \diff[2]{x}{t} \tag{Hooke's law, $\alpha=\mathrm{d}^2\theta/dt^2$} \\
	&=-m\omega^2x \tag{$\mathrm{d}^2x/dt^2=-\omega^2x$} \\
	k
	&=m \left( \frac{2\pi}{T} \right)^2 \tag{$\omega=2\pi/T$} \\
	\sqrt{\frac km}
	&=\frac{2\pi}{T} \\
	\Aboxed{T&=2\pi \sqrt{\frac mk}}
\end{align*}

\subsubsection{Period of Simple Pendulum}
Note that arc length $s=l\theta$ where $l$ is the length of the string. A mass is attached to the end of the string and allowed to oscillate. Acceleration is the second derivative of arc length. Figure 7.3.2.
\begin{align*}
    \Sigma F
	&=ma \\
	-mg\sin\theta
	&=m\cdot \diff[2]st \tag{Restoring force and $\alpha=\mathrm{d}^2s/dt^2$} \\
	-g\sin\theta
	&=l\cdot\diff[2]{\theta}{t} \tag{Arc length $s=l\theta$} \\ 
	&=-l\omega^2\theta \tag{$\mathrm{d}^2\theta/dt^2=-\omega^2\theta$} \\
	g\theta&=l\omega^2\theta \tag{$\sin\theta\approx\theta$} \\
	g&=l \left( \frac{2\pi}{T} \right)^2 \tag{$\omega=2\pi/T$} \\
	\Aboxed{T&=2\pi \sqrt{\frac lg}}
\end{align*}

\subsubsection{Period of a Physical Pendulum}

A flat physical object of mass $m$ is oscillating. Figure 7.3.3. Let $r$ be the distance from the axis of rotation to center of mass. Parallel axis theorem:
\begin{equation*}
	I'=I_\text{cm}+Mr^2
\end{equation*}
where $I'$ is the moment of inertia of the physical pendulum about the point of rotation.

Starting with Newton's second law for rotation:
\begin{align*}
	\Sigma \tau
	&=I' \alpha \\
	-rmg\sin\theta
	&=I'\alpha \tag{Restoring force}\\
	-rmg\theta
	&=I'\cdot \diff[2]{\theta}{t} \tag{$\alpha=\mathrm{d}^2\theta/dt^2,\sin\theta\approx\theta$} \\
	-\frac{rmg\theta}{I'}
	&=-\omega^2\theta \tag{$\mathrm{d}^2\theta/dt^2=-\omega^2\theta$}\\
	\omega^2
	&=\frac{rmg}{I'} \\
	\frac{2\pi}{T} \tag{$\omega=2\pi/T$}
	&=\sqrt{\frac{rmg}{I'}} \\
	\Aboxed{T&=2\pi \sqrt{\frac{I'}{rmg}}}
\end{align*}

\subsubsection{Torsional Pendulum}

The motion of a torsional pendulum is analogous to that of a spring-mass system. Figure 7.3.4.

Hooke's law for rotation: $\tau=-k\Delta \theta$ where $\theta$ is the restoring torque, $k$ is the torsional constant and $\Delta \theta$ is the twisting angle.
\begin{align*}
	\Sigma \tau 
	&=I\alpha \\
	-k\theta
	&=I\cdot \diff[2]{\theta}t \tag{$\Sigma\tau=-k\theta,\alpha=\mathrm{d}^2\theta/\mathrm{d}t^2$} \\
	-\frac{k\theta}{I}
	&=-\omega^2\theta \tag{$\mathrm{d}^2\theta/\mathrm{d}t^2=-\omega^2\theta$} \\
	\frac{2\pi}{T}
	&=\sqrt{\frac kI} \tag{$\omega=2\pi/T$} \\
	\Aboxed{T&=2\pi \sqrt{\frac Ik}}
\end{align*}
where $I$ is the moment of inertia of the hanging rotating mass about the center of mass, and $k$ is the torsion constant (equivalent of spring constant).

\section{Drag}

\definition{\textbf{Drag} is a friction force due to interactions between fluid (gas or liquid) molecules and the object.}

\begin{itemize}
	\item Drag force is often denoted $D, F_\text{d},F_\text{D}$.
	\item Drag force $F_\text{D}$ is dependent on the relative speed between the object and the fluid.
	\item $F_\text{D}$ \textit{imposes} a terminal velocity $v_\text{T}$ on the object, which occurs when the object moves long/fast enough.
\end{itemize}
Total mechanical energy is \ul{not conserved when there is drag.} Negative work is done by the friction and some energy is transferred to thermal energy.

How to solve drag problems:
\begin{enumerate}
	\item Draw free-body diagram and include positive direction.
	\item Write Newton's second law: $\Sigma F=F-F_\text{D}=ma=m(\mathrm{d}v/\mathrm{d}t)$. $F_\text{D}$ is dependent on velocity $v$ so this is a differential equation.
\end{enumerate}

\subsection{Relationship Between Velocity and Drag}

In small objects traveling at a low velocity, drag force is directly proportional to velocity (negative sign indicates $F_\text{D}$ acts opposite to direction of $v$):
\begin{align*}
	F_\text{D}&=-bv
	\intertext{In larger objects with greater velocity, drag is proportional to velocity squared:}
	F_\text{D}&=-kv^2
\end{align*}
In both equations, $b$ and $k$ are constants that depend on the object's size, shape/orientation, surface characteristics (smooth/rough), velocity/density of fluid etc.

If two objects $P$ and $Q$ look identical but $P$ is heavier, then $P$ will accelerate for a longer time before reaching terminal velocity $v_\text{T}$ because its weight force is greater. $F_\text{D}$ is independent of mass but acceleration is influenced.

Terminal velocity $v_\text{T}=\lim_{t\to\infty}v(t)$.

\subsection{Graphing}

When there is drag, position, velocity and acceleration are all \ul{exponential}. We drop an object with drag, and that positive is down:
\begin{itemize}
	\item Position will increase exponentially (concave up), tends to increasing linearly as it reaches terminal velocity.
	\item Velocity will increase exponentially (concave down) and flattens out as it reaches terminal velocity.
	\item Acceleration decreases exponentially (concave up) and tends to zero as it reaches terminal velocity.
\end{itemize}

In this case, the net force $\Sigma F=mg-kv=ma\implies mg-kv=m(\mathrm{d}v/\mathrm{d}t)$. This differential equation can be solved by separation of variables, yielding an exponential equation for $v$ with $v=0$ at $t=0$ (this value is obtained by using the correct value of $C$). Solving the differential equation is left as an exercise to the reader.

\section{Gravitational Theory}

\theorem{Newton's law of universal gravitation}{
	\begin{equation*}
		\lVert F_\text{g} \rVert 
		=\frac{Gm_1m_2}{r^2}
	\end{equation*}
	where
	\begin{align*}
		m&=\text{mass (kg)} \\
		r&=\text{distance between centers of mass (m)} \\
		F_\text{g}&=\text{force due to gravity (N)} \\
		G&=\text{gravitational constant (\si{\meter^3/(\kilogram\cdot\second^2)})} \\
		 &=6.67\cdot 10^{-11}\,\si{\frac{\meter^3}{\kilogram\cdot\second^2}}
	\end{align*}
	By Newton's third law, each object exerts a force on the other. The two forces are equal in magnitude but opposite in direction.
}

By Newton's second law:
\begin{equation*}
	F_\text{g}=ma=\frac{GMm}{r^2}
	\implies a=\frac{GM}{r^2}
\end{equation*}
is the acceleration due to gravity \ul{near the planet's surface:} $g=GM/r^2$. At higher distances from the surface, the distance $r$ is greater, so acceleration due to gravity is lower.

The relationship between the gravitational force exerted on an object $F_\text{g}$ and its center of mass's distance to the planet's center of mass $r$ is in Figure 8.1. $F_\text{g}$ is highest on the surface of the planet. \ul{From $0$ to the surface, $F_\text{g}$ increases linearly; for higher values, $F_\text{g}$ decreases proportional to $1/r^2$.}

\subsection{Orbits}

For high values of $r$ and $h$ (in Figure 8.2), $r\approx h$. Combining centripetal force and gravitational force:
\begin{equation*}
	F_\text{g}=\frac{GMm}{r^2}=\frac{mv^2}{r}=F_\text{c}
	\implies v=\sqrt{\frac{GM}{r}}
\end{equation*}
We can also derive Kepler's law:
\begin{equation*}
    \frac{r^3}{T^2}=\frac{GM}{4\pi^2}
\end{equation*}

\subsection{Potential Energy}

Recall the relationship between potential energy and force:
\begin{equation*}
    F=-\diff Ux \\
	\Delta U=-\int F\,\mathrm{d}x
\end{equation*}
We have force due to gravity $F_\text{g}$ (negative sign indicates an attractive force):
\begin{align*}
	F_\text{g}&=-\frac{GMm}{r^2} \\
	\Delta U_\text{g}&=-\int \left( -\frac{GMm}{r^2} \right)\mathrm{d}r
	=-\frac{GMm}{r}
\end{align*}

\theorem{General potential energy due to gravitation}{
	The potential energy of a system of two objects from gravitation is:
	\begin{equation*}
		U_\text{g}=-\frac{GMm}{r}
	\end{equation*}
	where:
	\begin{align*}
		U_\text{g}&=\text{potential energy ($\si{\kilogram\cdot\meter^2\per\second^2}=\si{\joule}$)} \\
		M&=\text{mass of larger object, like the Earth (kg)} \\
		m&=\text{mass of smaller object (kg)} \\
		r&=\text{distance between centers of mass (m)} \\
		G&=\text{gravitational constant (\si{\meter^3/(\kilogram\cdot\second^2)})} \\
		 &=6.67\cdot 10^{-11}\,\si{\frac{\meter^3}{\kilogram\cdot\second^2}}
	\end{align*}
}

The gravitational potential energy of a system with three objects is the sum of the gravitational potential energies of any pair of two objects:
\begin{equation*}
	U_{\text{g},\text{system}}=U_{\text{g},1,2}+U_{\text{g},1,3}+U_{\text{g},2,3}
\end{equation*}
The total energy of a object orbiting another, denoted $E_\text{T}$, is the sum of its kinetic and (gravitational) potential energies. 

\ul{In a circular orbit}, the velocity of the orbiting object is $v=\sqrt{GM/r}$ (can be found with $F_\text{c}=mv^2/r=GMm/r^2=F_\text{g}$).
\begin{equation*}
	E_\text{T}
	=K+U_\text{g}=\frac12mv^2+\left( -\frac{GMm}{r} \right)
	=\frac12m \left( \sqrt{\frac{GM}{r}} \right)^2-\frac{GMm}{r}
	=\frac12 \frac{GMm}{r}-\frac{GMm}{r}
	=-\frac12 \frac{GMm}{r}
	=\frac12U_\text{g}
\end{equation*}

\theorem{Total energy of orbiting object}{
	\ul{In circular orbits only}, the total energy $E_\text{T}$ is exactly half the gravitational potential energy.
	\begin{equation*}
		E_\text{T}
		=-\frac12 \frac{GMm}{r}
		=\frac12 U_\text{g}
	\end{equation*}
}

See Figure 9.1 for graph of potential energy with $U_\text{g}=mgh$ and $U_\text{g}=-GMm/r$.

\subsection{Elliptical Orbits}

Like with circular orbits, $m$ and $M$ are constant. However, the distance between centers of mass $r$ is \textbf{not} constant in an elliptical orbit, so its speed $\lVert v \rVert$, kinetic energy $K=mv^2/2$, and gravitational potential energy $U_\text{g}=-GMm/r$ are also not constant (they are constant for circular orbits). Angular momentum $L=\lVert r\times p \rVert =mvr\sin\theta$ and total energy $E_\text{T}$ are constant for both circular and elliptical orbits. In both, total energy $E_\text{T}<0$.
\begin{alignat*}{2}
	M&=\text{constant} &\qquad M&=\text{constant} \\
	m&=\text{constant} &\qquad m&=\text{constant} \\
	r&=\text{constant} &\qquad \Aboxed{r&=\text{not constant}} \\
	\lVert v \rVert &=\text{constant} &\qquad \Aboxed{\lVert v \rVert &=\text{not constant}} \\
	K&=\text{constant} &\qquad \Aboxed{K&=\text{not constant}} \\
	U_\text{g}&=\text{constant} &\qquad \Aboxed{U_\text{g}&=\text{not constant}} \\
	E_\text{T}&=\text{constant} &\qquad E_\text{T}&=\text{constant} \\
	L&=\text{constant} &\qquad L&=\text{constant}
\end{alignat*}

\subsection{Escape Velocity}

\definition{The \textbf{escape velocity} is the speed at which an object ``escapes'' the gravitational field.}

For now, we assume the following:
\begin{itemize}
	\item The rocket is launched straight upward (they are not).
	\item No propulsion along the way (there is in an actual rocket).
	\item At ``infinity'' the rocket is at rest: $v=K_\text{rocket}=0$. The rocket slows down as it travels.
	\item Negligible drag.
	\item There are no gravitational effects on the rocket as it travels (there always will be).
\end{itemize}
These assumptions do not hold in reality, so the escape velocity we calculate is highly theoretical.

By conservation of energy:
\begin{align*}
	K_\text{i}+U_\text{g,i}
	&=K_\text{f}+U_\text{g,f}
	\intertext{The ``final'' energies are taken at infinity, where there is no velocity and no effect from gravity.}
	\frac12mv_0^2-\frac{GMm}{r}&=0 \\
	v_0&=\sqrt{\frac{2GM}{r}}
	\intertext{At a certain distance to the center of the Earth $R$, the escape velocity $v_\text{esc}$ is:}
	v_\text{esc}&=\sqrt{\frac{2GM}{R}}
\end{align*}

\theorem{Escape velocity}{
	The escape velocity when orbiting an object of mass $M$ at a distance $R$ from its center of mass is:
	\begin{equation*}
		v_\text{esc}=\sqrt{\frac{2GM}{R}}
	\end{equation*}
}

\subsection{Orbital Dynamics}

Hohmann transfer orbits: moving object frm lower circular orbit to higher elliptical orbit, to higher circular orbit (intercepting with moon, for example).

\subsection{Newton's Shell Theorem}

Relationship between gravitational force and distance to the center of the Earth: increases linearly when under the surface, decreases proportionally to $1/r^2$ when outside. Why?

Break down the Earth into infinitely many point masses, each with mass $\mathrm{d}m$. An object of mass $m$ outside the Earth experiences attraction to each of these point masses in that direction and with magnitude $\mathrm{d}F_\text{g}$. Each such attraction to each point mass will cancel each other out, and the final net force is one pointing toward the center of mass. See Figure 9.2. The gravitational force derived from parts of the Earth outside the dotted line will cancel out with each other, and the net force is zero.

When inside the Earth, and for an object inside the Earth, only those parts of the Earth with distance to the center (radius) \ul{less} than that of the object, will exert a net force of attraction. See Figure 9.3.

\theorem{Density}{
	\begin{equation*}
	    \rho=\frac{M}{V}
	\end{equation*}
	where:
	\begin{align*}
		\rho&=\text{density (\si{\kilogram/\meter^3})} \\
		M&=\text{mass (kg)} \\
		V&=\text{volume (\si{m^3})}
	\end{align*}
}

We can derive the force of gravitation while inside the earth. Assume the Earth is of uniform density, then:
\begin{equation*}
	\rho_\text{inside}=\frac{M_\text{inside}}{V_\text{inside}}=\frac{M_E}{V_E}=\rho_E
\end{equation*}
Since we know the volume of a sphere $V=(4/3)\pi r^3$:
\begin{equation*}
	\frac{M_\text{inside}}{\frac43\pi R_\text{inside}^3}=\frac{M_E}{\frac43\pi R_E^3}
	\implies M_\text{inside}=\frac{R_\text{inside}^3}{R_E^3}M_E
\end{equation*}
From $F_\text{g}=GMm/r^2$, where $m$ is the mass of the object, and $r$ is its distance to the center. Note that the object's distance $r$ is the same as the ``inside radius'' $R_\text{inside}$:
\begin{equation*}
	F_\text{g}=\frac{GM_\text{inside}m}{r^2}=\frac{Gm}{r^2}M_\text{inside}
	=\frac{Gm}{r^2}\frac{R_\text{inside}^3}{R_E^3}M_E
	=\frac{GM_Emr^3}{R_E^3r^2}
	=\underbrace{\left(\frac{GM_Em}{R_E^3}\right)}_\text{constant}r
\end{equation*}
Thus the force and acceleration are proportional to $r$.

\theorem{Newton's shell theorem}{
	The net gravitational force exerted on an object $m$ outside a specifically symmetric object $M$ can be determined by treating $M$ as a point mass located at its center of mass.
}

\end{document}

