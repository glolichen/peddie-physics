\documentclass{article}
\usepackage[letterpaper,portrait,top=0.4in, left=0.6in, right=0.6in, bottom=1in]{geometry}

\usepackage{amsmath, amsfonts, amsthm, amssymb}
\usepackage{soul}
\usepackage{graphicx, float}
\usepackage{suffix}
\usepackage{soul}
\usepackage{esdiff}
\usepackage{multicol}
\usepackage{cancel}
\usepackage{mdframed}
\usepackage{mathtools}
\usepackage{tcolorbox}
\usepackage[colorlinks, linkcolor=blue]{hyperref}
\usepackage[per-mode=symbol]{siunitx}
\usepackage{setspace}
\usepackage{parskip}
\usepackage{enumitem}
\usepackage{titling}
\usepackage{mlmodern}

\newcommand{\alignedintertext}[1]{%
  \noalign{%
    \vtop{\hsize=\linewidth#1\par
    \expandafter}%
    \expandafter\prevdepth\the\prevdepth
  }%
}

\newcommand{\definition}[1]{\begin{tcolorbox}[colback=red!5!white,colframe=red!75!black,parbox=false] #1 \end{tcolorbox}}

\newcommand{\theorem}[2]{\begin{tcolorbox}[title={#1},colback=blue!5!white,colframe=blue!75!black,parbox=false] #2 \end{tcolorbox}}
\WithSuffix\newcommand\theorem*[1]{\begin{tcolorbox}[colback=blue!5!white,colframe=blue!75!black,parbox=false] #1 \end{tcolorbox}}

\newcommand{\example}[2]{\begin{tcolorbox}[title={Example: #1},colback=brown!5!white,colframe=brown!75!black,parbox=false] #2 \end{tcolorbox}}

\newcommand{\remark}[2]{\begin{tcolorbox}[title={#1},colback=black!5!white,colframe=black!75!black,parbox=false] #2 \end{tcolorbox}}
\WithSuffix\newcommand\remark*[1]{\begin{tcolorbox}[colback=black!5!white,colframe=black!75!black,parbox=false] #1 \end{tcolorbox}}

\newcommand{\hlc}[2][yellow]{\sethlcolor{#1}\hl{#2}}

\newcommand*{\deriv}[1][x]{\ensuremath{\dfrac{\mathrm{d}}{\mathrm{d}#1}}}
\newcommand*{\floor}[1]{\ensuremath{\lfloor #1\rfloor}}

\title{\vspace*{-40pt}\textsc{AP Physics C: Electricity and Magnetism Notes}}
\author{Jayden Li}
\date{Winter and Spring Trimesters, 2026}

\begin{document}
\setstretch{1.25}
\fontsize{11pt}{12pt}\selectfont
\setlength{\abovedisplayskip}{\abovedisplayskip/2}
\setlength{\belowdisplayskip}{\belowdisplayskip/2}
\setlength{\parindent}{0pt}
\setlength{\parskip}{2ex plus 0.5ex minus 0.2ex}
\maketitle

\tableofcontents
% \newpage

\section{Circuit}

Equipment:
\begin{itemize}
	\item D-cell battery/cells: with positive and negative terminals.
	\item Battery holder.
	\item Ammeter (machine labeled ``\ul{A}'').
	\item Multimeter (can be ammeter, voltmeter or ohmmeter): has ``test leads'' or ``probes.''
	\item Single pole single throw switch and single pole double throw switch.
	\item Wires: alligator-alligator wires (alligator clips on both ends), alligator-banana wires, banana-banana wires.
	\item Sockets (light bulbs screw into the hole).
	\item Round bulbs and long bulbs.
\end{itemize}

There are electrons everywhere in the circuit (all metal solids have delocalized electrons). The battery serves as a ``pump'' to move electrons around through the circuit. We can use the voltmeter in the multimeter to measure the ``pump strength.'' 

We can find the potential difference across the battery $\Delta V_\text{battery}$ and across the two light bulbs $\Delta V_{\text{B}_1}$ and $\Delta V_{\text{B}_2}$.
\begin{align*}
	\Delta V_\text{battery}&=4.34\,\si{\volt} \\
	\Delta V_{\text{B}_1}&=1.84\,\si{\volt} \\
	\Delta V_{\text{B}_2}&=1.83\,\si{\volt}
\end{align*}
Since energy is conserved the potential difference of the bulbs should be the same as that of the battery, but it is not. We need to take $\Delta V$ of the wires and switch into account as well. 

\definition{
	\textbf{Potential difference} ($\Delta V$) is measured in volts (V), and is related to energy. In Ye Olde Times, it is denoted $V$ or emf (electromotive force), and $\Delta V_\text{bat}$ is denoted $\varepsilon$.

	\textbf{Flow rate/current} ($I$) measures the number of electrons that pass a certain point per unit of time, and is measured in amperes (A).

	\textbf{Resistance} ($R$) is how difficult it is for electrons to flow through and is measured in ohms ($\Omega$).
}

The current is equal across the circuit.

\theorem{Ohm's Law}{
	\begin{equation*}
	    \Delta V=IR
	\end{equation*}
	where:
	\begin{align*}
		\Delta V&=\text{potential difference (V)} \\
		I&=\text{current (A)} \\
		R&=\text{resistance (\si{\ohm})}
	\end{align*}
}

Resistors convert electrical energy to thermal energy. Light bulbs are a special case of resistor that convert electrical energy to thermal and light energy.

\theorem{Resistors in series}{
	When resistors are connected in series, current is equal across any part of the circuit:
	\begin{align*}
		I_\text{bat}&=I_1=I_2=I_3 \\
		\intertext{Assuming no potential is lost anywhere:}
		\Delta V_\text{bat}&=\Delta V_1+\Delta V_2+\Delta V_3 \\
		R_\text{total}&=R_1+R_2+R_3
	\end{align*}
}

\theorem{Resistance}{
	\begin{equation*}
	    R=\frac{\rho l}{A}
	\end{equation*}
	where $\rho$ is the resistivity of the material (\si{\ohm\meter}), $l$ is length (m) and $A$ is cross-sectional area (\si{\meter^2}).
}

\definition{A \textbf{junction/node} is a point where three or more wires are connected.

A \textbf{branch} connects two nodes.}

To connect a parallel circuit, multiple junctions need to be made. 

\theorem{Resistors in parallel}{
	The currents are different at different parts of the circuit:
	\begin{align*}
		I_\text{bat}&=I_1+I_2+I_3 \\
		\Delta V_\text{bat}&=\Delta V_1=\Delta V_2=\Delta V_3 \\
		\frac{1}{R_\text{total}}&=\frac{1}{R_1}+\frac{1}{R_2}+\frac{1}{R_3}
	\end{align*}
	Similar to springs in parallel.

	A shortcut: if the total equivalent resistance of a parallel circuit is $R_T$ and the current entering the circuit is $I_T$, the current through branch $i$ is. Since the circuit is parallel $R_T$ is less than the individual $R_i$'s.
	\begin{equation*}
	    I_i
		=I_T\cdot \frac{\frac{1}{R_i}}{\sum_k \frac{1}{R_k}}
		=I_T\cdot \frac{\frac{1}{R_i}}{\frac{1}{R_T}}
		=I_T\cdot \frac{R_T}{R_i}
	\end{equation*}
}
From $I=\Delta V/R$ and $\Delta V_\text{bat}=\Delta V_1=\Delta V_2=\Delta V_3$:
\begin{equation*}
	I_\text{bat}=I_1+I_2+I_3
	\implies \frac{\Delta V_\text{bat}}{R_\text{total}}=\frac{\Delta V_1}{R_1}+\frac{\Delta V_2}{R_2}+\frac{\Delta V_3}{R_3}
	\implies \boxed{\frac{1}{R_\text{total}}=\frac{1}{R_1}+\frac{1}{R_2}+\frac{1}{R_3}}
\end{equation*}

\remark{Electron Sea Model}{
	In metals, atoms form metallic bonds: each metal atom releases its valence electrons, which then form a sea of delocalized electrons. These electrons are then attracted to multiple atoms. These bonds are very strong and lead to high melting points for metals.

	Delocalized electrons can move in a direction caused by an electric force.
}

Electrons flow from the negative terminal to the positive terminal. Conventional current is the direction in which positive charges flow, and is opposite to the direction electrons flow. The direction electrons flow is marked $e^-$ and conventional current is marked $I$. The ``long'' end of the battery symbol is positive.

Ohmmeters are connected when there are no charges flowing; voltmeter and ammeter need the circuit to be active. Ammeter is connected in a circuit; voltmeter and ohmmeter measure across a circuit component (with a multimeter). In reality, these devices will modify the circuit when they are attached. An ammeter actually has a very small resistance, but we assume it has \ul{zero resistance}. Also assume that a voltmeter has \ul{infinite resistance}.

If two components are on different paths to the battery, they have the same potential difference (parallel).

\definition{A \textbf{combination circuit} is one which is not purely series or purely parallel. We can analyze such a circuit by calculating the total/equivalent resistance.}

\theorem{Power}{
	\begin{equation*}
	    P
		=I\Delta V
		=I^2R
		=\frac{\Delta V^2}{R}
	\end{equation*}
	where:
	\begin{align*}
		P&=\text{power (W)} \\
		I&=\text{current (A)} \\
		\Delta V&=\text{potential difference (V)} \\
		R&=\text{resistance (\si{\ohm})}
	\end{align*}
	The sum of power in each resistors equals the power in the battery.
}

Resistors dissipate energy by converting electrical energy to thermal energy at a rate equal to $P=I\Delta V=\ldots$

\remark{Light bulb ratings}{
	Suppose a light bulb is rated at $p\,\si{\watt}$ and $v\,\si{\volt}$. $p$ relates to the brightness of the light bulb. From the formula $P=(\Delta V)^2/R$, we can determine the resistance of the light bulb: $R=v^2/p$.

	Light bulb assumptions:
	\begin{itemize}
	    \item The brightness of a light bulb is directly proportional to the power.
		\item The resistance of a light bulb is constant.
	\end{itemize}
}

\remark{Electric bills}{
	The house is billed by energy used in kWh, which is actually a unit of energy.
	\begin{equation*}
		1\,\text{kWh}=3.6\times10^6\,\si{\joule}
	\end{equation*}
}

In a short circuit, a path with small/virtually zero resistance forms, adding a parallel path. The overall/equivalent resistance drops even though $\Delta V_\text{bat}$ is constant, increasing the current and therefore the power. A circuit breaker/fuse will trip/melt and open the circuit when current is higher than a certain limit.

\subsection{Internal Resistance}

In an ideal battery, there is no resistance: $R=0$. In practice, a battery contains chemical reactions to generate a potential difference/current, and as the products of the reaction increase, the internal resistance increase.

In a circuit diagram, a resistor is placed to either side of the symbol for a bettery. Sometimes, a dotted line includes these components. The potential difference of the battery and its internal resistance is $\Delta V_\text{T}$ (V).
\theorem{Potential difference and internal resistance of battery}{
	\begin{equation*}
		\Delta V_\text{T}=\varepsilon-Ir
	\end{equation*}
	where:
	\begin{align*}
		\Delta V_\text{T}&=\text{terminal potential difference (includes internal resistance) (V)} \\
		\varepsilon&=\text{potential difference of ideal battery (V)} \\
		I&=\text{current through battery (A)} \\
		r&=\text{internal resistance ($\Omega$)}
	\end{align*}
}

When batteries are connected in series, the ideal potential difference ($\varepsilon$) increases, but internal resistance also increases. The actual increase in terminal potential difference ($\Delta V_\text{T}$) might be quite low, and a lot of thermal energy is produced.

\ul{If batteries are connected in parallel}, there is no increase in potential difference. But the current in the branches where batteries are connected is reduced as batteries are added in parallel, which reduces internal resistance and $\Delta V$ due to internal resistance $Ir$. The battery also lasts longer.

\ul{Potential difference opposing each other}: positive terminals are adjacent to each other. The battery with a lower potential difference is being ``charged,'' until the batteries have the same potential difference.

\theorem{Kirchhoff's Laws}{
	\begin{enumerate}
		\item Junction/node law (law of conservation of charges): at any junction (place where 3 or more wires are connected together), the sum of all currents flowing into the junction equals the sum of all currents flowing out of the junction. Charges are conserved.
		\item Loop law: around \ul{any} loop in the circuit, the sum of all $\Delta V$ changes must equal zero. In conventional current: the battery produces a positive $\Delta V$, resistors and light bulbs produce a negative $\Delta V$. The sum of all components equals $0$. Opposite of using flow of electrons as direction.
	\end{enumerate}
}

\end{document}

