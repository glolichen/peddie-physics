\documentclass{article}
\usepackage[letterpaper,portrait,top=0.4in, left=0.6in, right=0.6in, bottom=1in]{geometry}

\usepackage{amsmath, amsfonts, amsthm, amssymb}
\usepackage{soul}
\usepackage{graphicx, float}
\usepackage{suffix}
\usepackage{soul}
\usepackage{esdiff}
\usepackage{multicol}
\usepackage{cancel}
\usepackage{mdframed}
\usepackage{mathtools}
\usepackage{tcolorbox}
\usepackage[colorlinks, linkcolor=blue]{hyperref}
\usepackage[per-mode=symbol]{siunitx}
\usepackage{setspace}
\usepackage{parskip}
\usepackage{enumitem}
\usepackage{titling}
\usepackage{mlmodern}

\newcommand{\alignedintertext}[1]{%
  \noalign{%
    \vtop{\hsize=\linewidth#1\par
    \expandafter}%
    \expandafter\prevdepth\the\prevdepth
  }%
}

\newcommand{\definition}[1]{\begin{tcolorbox}[colback=red!5!white,colframe=red!75!black,parbox=false] #1 \end{tcolorbox}}

\newcommand{\theorem}[2]{\begin{tcolorbox}[title={#1},colback=blue!5!white,colframe=blue!75!black,parbox=false] #2 \end{tcolorbox}}
\WithSuffix\newcommand\theorem*[1]{\begin{tcolorbox}[colback=blue!5!white,colframe=blue!75!black,parbox=false] #1 \end{tcolorbox}}

\newcommand{\defitheorem}[2]{\begin{tcolorbox}[title={#1},colback=violet!5!white,colframe=violet!75!black,parbox=false] #2 \end{tcolorbox}}
\WithSuffix\newcommand\defitheorem*[1]{\begin{tcolorbox}[colback=violet!5!white,colframe=violet!75!black,parbox=false] #1 \end{tcolorbox}}

\newcommand{\example}[2]{\begin{tcolorbox}[title={Example: #1},colback=brown!5!white,colframe=brown!75!black,parbox=false] #2 \end{tcolorbox}}

\newcommand{\remark}[2]{\begin{tcolorbox}[title={#1},colback=black!5!white,colframe=black!75!black,parbox=false] #2 \end{tcolorbox}}
\WithSuffix\newcommand\remark*[1]{\begin{tcolorbox}[colback=black!5!white,colframe=black!75!black,parbox=false] #1 \end{tcolorbox}}

\newcommand{\hlc}[2][yellow]{\sethlcolor{#1}\hl{#2}}

\newcommand*{\deriv}[1][x]{\ensuremath{\dfrac{\mathrm{d}}{\mathrm{d}#1}}}
\newcommand*{\floor}[1]{\ensuremath{\lfloor #1\rfloor}}

\title{\vspace*{-40pt}\textsc{AP Physics C: Electricity and Magnetism Notes}}
\author{Jayden Li}
\date{Winter and Spring Trimesters, 2026}

\begin{document}
\setstretch{1.25}
\fontsize{11pt}{12pt}\selectfont
\setlength{\abovedisplayskip}{\abovedisplayskip/2}
\setlength{\belowdisplayskip}{\belowdisplayskip/2}
\setlength{\parindent}{0pt}
\setlength{\parskip}{2ex plus 0.5ex minus 0.2ex}
\maketitle

\tableofcontents
% \newpage

\section{Circuits}

Equipment:
\begin{itemize}
	\item D-cell battery/cells: with positive and negative terminals.
	\item Battery holder.
	\item Ammeter (machine labeled ``\ul{A}'').
	\item Multimeter (can be ammeter, voltmeter or ohmmeter): has ``test leads'' or ``probes.''
	\item Single pole single throw switch and single pole double throw switch.
	\item Wires: alligator-alligator wires (alligator clips on both ends), alligator-banana wires, banana-banana wires.
	\item Sockets (light bulbs screw into the hole).
	\item Round bulbs and long bulbs.
\end{itemize}

There are electrons everywhere in the circuit (all metal solids have delocalized electrons). The battery serves as a ``pump'' to move electrons around through the circuit. We can use the voltmeter in the multimeter to measure the ``pump strength.'' 

We can find the potential difference across the battery $\Delta V_\text{battery}$ and across the two light bulbs $\Delta V_{\text{B}_1}$ and $\Delta V_{\text{B}_2}$.
\begin{align*}
	\Delta V_\text{battery}&=4.34\,\si{\volt} \\
	\Delta V_{\text{B}_1}&=1.84\,\si{\volt} \\
	\Delta V_{\text{B}_2}&=1.83\,\si{\volt}
\end{align*}
Since energy is conserved the potential difference of the bulbs should be the same as that of the battery, but it is not. We need to take $\Delta V$ of the wires and switch into account as well. 

\definition{
	\textbf{Potential difference} ($\Delta V$) is measured in volts (V), and is related to energy. In Ye Olde Times, it is denoted $V$ or emf (electromotive force), and $\Delta V_\text{bat}$ is denoted $\varepsilon$.

	\textbf{Flow rate/current} ($I$) measures the number of electrons that pass a certain point per unit of time, and is measured in amperes (A).

	\textbf{Resistance} ($R$) is how difficult it is for electrons to flow through and is measured in ohms ($\Omega$).
}

The current is equal across the circuit.

\theorem{Ohm's Law}{
	\begin{equation*}
	    \Delta V=IR
	\end{equation*}
	where:
	\begin{align*}
		\Delta V&=\text{potential difference (V)} \\
		I&=\text{current (A)} \\
		R&=\text{resistance (\si{\ohm})}
	\end{align*}
}

Resistors convert electrical energy to thermal energy. Light bulbs are a special case of resistor that convert electrical energy to thermal and light energy.

\theorem{Resistors in series}{
	When resistors are connected in series, current is equal across any part of the circuit:
	\begin{align*}
		I_\text{bat}&=I_1=I_2=I_3 \\
		\intertext{Assuming no potential is lost anywhere:}
		\Delta V_\text{bat}&=\Delta V_1+\Delta V_2+\Delta V_3 \\
		R_\text{total}&=R_1+R_2+R_3
	\end{align*}
}

\theorem{Resistance}{
	\begin{equation*}
	    R=\frac{\rho l}{A}
	\end{equation*}
	where:
	\begin{align*}
	    \rho&=\text{resistivity of material (\si{\ohm\meter})} \\
		l&=\text{length (m)} \\
		A&=\text{cross-sectional area (\si{m^2})}
	\end{align*}
}

\definition{A \textbf{junction/node} is a point where three or more wires are connected.

A \textbf{branch} connects two nodes.}

To connect a parallel circuit, multiple junctions need to be made. 

\theorem{Resistors in parallel}{
	The currents are different at different parts of the circuit:
	\begin{align*}
		I_\text{bat}&=I_1+I_2+I_3 \\
		\Delta V_\text{bat}&=\Delta V_1=\Delta V_2=\Delta V_3 \\
		\frac{1}{R_\text{total}}&=\frac{1}{R_1}+\frac{1}{R_2}+\frac{1}{R_3}
	\end{align*}
	Similar to springs in parallel.

	A shortcut: if the total equivalent resistance of a parallel circuit is $R_T$ and the current entering the circuit is $I_T$, the current through branch $i$ is. Since the circuit is parallel $R_T$ is less than the individual $R_i$'s.
	\begin{equation*}
	    I_i
		=I_T\cdot \frac{\frac{1}{R_i}}{\sum_k \frac{1}{R_k}}
		=I_T\cdot \frac{\frac{1}{R_i}}{\frac{1}{R_T}}
		=I_T\cdot \frac{R_T}{R_i}
	\end{equation*}
}
From $I=\Delta V/R$ and $\Delta V_\text{bat}=\Delta V_1=\Delta V_2=\Delta V_3$:
\begin{equation*}
	I_\text{bat}=I_1+I_2+I_3
	\implies \frac{\Delta V_\text{bat}}{R_\text{total}}=\frac{\Delta V_1}{R_1}+\frac{\Delta V_2}{R_2}+\frac{\Delta V_3}{R_3}
	\implies \boxed{\frac{1}{R_\text{total}}=\frac{1}{R_1}+\frac{1}{R_2}+\frac{1}{R_3}}
\end{equation*}

\remark{Electron Sea Model}{
	In metals, atoms form metallic bonds: each metal atom releases its valence electrons, which then form a sea of delocalized electrons. These electrons are then attracted to multiple atoms. These bonds are very strong and lead to high melting points for metals.

	Delocalized electrons can move in a direction caused by an electric force.
}

Electrons flow from the negative terminal to the positive terminal. Conventional current is the direction in which positive charges flow, and is opposite to the direction electrons flow. The direction electrons flow is marked $e^-$ and conventional current is marked $I$. The ``long'' end of the battery symbol is positive.

Ohmmeters are connected when there are no charges flowing; voltmeter and ammeter need the circuit to be active. Ammeter is connected in a circuit; voltmeter and ohmmeter measure across a circuit component (with a multimeter). In reality, these devices will modify the circuit when they are attached. An ammeter actually has a very small resistance, but we assume it has \ul{zero resistance}. Also assume that a voltmeter has \ul{infinite resistance}.

If two components are on different paths to the battery, they have the same potential difference (parallel).

\definition{A \textbf{combination circuit} is one which is not purely series or purely parallel. We can analyze such a circuit by calculating the total/equivalent resistance.}

\defitheorem{Power}{
	\begin{equation*}
	    P
		=I\Delta V
		=I^2R
		=\frac{\Delta V^2}{R}
	\end{equation*}
	where:
	\begin{align*}
		P&=\text{power (W)} \\
		I&=\text{current (A)} \\
		\Delta V&=\text{potential difference (V)} \\
		R&=\text{resistance (\si{\ohm})}
	\end{align*}
	The sum of power in each resistors equals the power in the battery.
}

Resistors dissipate energy by converting electrical energy to thermal energy at a rate equal to $P=I\Delta V=\ldots$

\remark{Light bulb ratings}{
	Suppose a light bulb is rated at $p\,\si{\watt}$ and $v\,\si{\volt}$. $p$ relates to the brightness of the light bulb. From the formula $P=(\Delta V)^2/R$, we can determine the resistance of the light bulb: $R=v^2/p$.

	Light bulb assumptions:
	\begin{itemize}
	    \item The brightness of a light bulb is directly proportional to the power.
		\item The resistance of a light bulb is constant.
	\end{itemize}
}

\remark{Electric bills}{
	The house is billed by energy used in kWh, which is actually a unit of energy.
	\begin{equation*}
		1\,\text{kWh}=3.6\times 10^6\,\si{\joule}
	\end{equation*}
}

In a short circuit, a path with small/virtually zero resistance forms, adding a parallel path. The overall/equivalent resistance drops even though $\Delta V_\text{bat}$ is constant, increasing the current and therefore the power. A circuit breaker/fuse will trip/melt and open the circuit when current is higher than a certain limit.

\subsection{Internal Resistance}

In an ideal battery, there is no resistance: $R=0$. In practice, a battery contains chemical reactions to generate a potential difference/current, and as the products of the reaction increase, the internal resistance increase.

In a circuit diagram, a resistor is placed to either side of the symbol for a battery. Sometimes, a dotted line includes these components. The potential difference of the battery and its internal resistance is $\Delta V_\text{T}$ (V).
\theorem{Potential difference and internal resistance of battery}{
	\begin{equation*}
		\Delta V_\text{T}=\varepsilon-Ir
	\end{equation*}
	where:
	\begin{align*}
		\Delta V_\text{T}&=\text{terminal potential difference (includes internal resistance) (V)} \\
		\varepsilon&=\text{potential difference of ideal battery (V)} \\
		I&=\text{current through battery (A)} \\
		r&=\text{internal resistance ($\Omega$)}
	\end{align*}
}

When batteries are connected in series, the ideal potential difference ($\varepsilon$) increases, but internal resistance also increases. The actual increase in terminal potential difference ($\Delta V_\text{T}$) might be quite low, and a lot of thermal energy is produced.

\ul{If batteries are connected in parallel}, there is no increase in potential difference. But the current in the branches where batteries are connected is reduced as batteries are added in parallel, which reduces internal resistance and $\Delta V$ due to internal resistance $Ir$. The battery also lasts longer.

\ul{Potential difference opposing each other}: positive terminals are adjacent to each other. The battery with a lower potential difference is being ``charged,'' until the batteries have the same potential difference.

\theorem{Kirchhoff's Laws}{
	\begin{enumerate}
		\item Junction/node law (law of conservation of charges): at any junction (place where 3 or more wires are connected together), the sum of all currents flowing into the junction equals the sum of all currents flowing out of the junction. Charges are conserved.
		\item Loop law: around \ul{any} loop in the circuit, the sum of all $\Delta V$ changes must equal zero. In conventional current: the battery produces a positive $\Delta V$, resistors and light bulbs produce a negative $\Delta V$. The sum of all components equals $0$. Opposite of using flow of electrons as direction.
	\end{enumerate}
}

\definition{The \textbf{ground} is where potential is zero volts. It is an ideal infinite source or sink of charge that can absorb unlimited charge without changing its potential of zero. It has both positive and negative charges.}

\subsection{Ideal Assumptions}

\begin{itemize}
	\item An ideal wire has \ul{zero} resistance.
	\item An ideal ammeter has \ul{zero} resistance.
	\item An ideal voltmeter has \ul{infinite} resistance.
	\item An ideal battery has \ul{zero} internal resistance.
	\item An ideal resistor is ohmic: it follows Ohm's law in that potential difference ($\Delta V$) is directly proportional to current ($I$), with the slope being resistance $R$.

	A light bulb is \ul{not ohmic}: its temperature will increase, increasing its resistance.
\end{itemize}

\subsection{Charge}

Current is the flow of charged particles, the charge ($q,Q$) being measured in coulombs (C). One coulomb is the amount of charge delivered by a current of one ampere per second.
\begin{equation*}
	\left[ \si{\ampere}=\frac{\si{\coulomb}}{\si{\second}} \right]
	\iff
	\left[ \si{\coulomb}=\si{\ampere\cdot\second} \right]
\end{equation*}
An electron has a charge of $-1.6\times 10^{-19}\,\si{\coulomb}$, and the charge of a proton has the same magnitude but opposite sign: $+1.6\times 10^{-19}\,\si{\coulomb}$.

\theorem{Charge and current}{
	\begin{equation*}
	    I=\frac{Q}{t}=\diff qt
	\end{equation*}
	where:
	\begin{align*}
		I&=\text{current (A)} \\
		Q&=\text{charge (C)} \\
		t&=\text{time (s)}
	\end{align*}
	This is helpful in non-steady state circuits where current is not constant and changes over time.
}

\section{Electrostatics}

\theorem{Coulomb's law}{
	\begin{equation*}
	    F_E=\frac{kq_1q_2}{r^2}
		\qquad k=\frac{1}{4\pi \varepsilon_0}
	\end{equation*}
	where:
	\begin{align*}
		F&=\text{electrostatic force (N)} \\
		k&=\text{Coulomb constant} \\
		 &=8.99\times 10^9\,\si{\newton\meter^2/\coulomb^2} \\
		\varepsilon_0&=\text{permittivity of free space (\si{\coulomb^2/\newton\meter^2})} \\
		&=8.85\times 10^{-12}\,\si{\coulomb^2/\newton\meter^2} \\
		q_1,q_2&=\text{charge (C)} \\
		r&=\text{distance between particles (m)}
	\end{align*}
}

There are positive and negative charges. The magnitude of the force is:
\begin{equation*}
	\lVert F_E \rVert =\frac{k \left|q_1\right| \left|q_2\right|}{r^2}
\end{equation*}
from which the direction of the force can be figured out by logic: opposite charges attract.

\definition{A \textbf{Van de Graaff generator} generates electrostatic energy by using a belt to transport electrons to a spherical dome. Electrons try to spread out as much as possible and are position on the dome's surface.}

\theorem{Constants (on equation sheet)}{
	The mass of an electron is:
	\begin{align*}
		m_e&=9.11\times 10^{-31}\,\si{\kilo\gram}
		\intertext{The elementary charge is the smallest possible charge, and is equal to the magnitude of the charge of a proton or electron.}
		Q_e&=1.60\times 10^{-19}\,\si{\coulomb}
		\intertext{The gravitational constant from Mechanics is:}
		G&=6.67\times 10^{-11}\,\si{\newton\meter^2\per\kilo\gram^2}
	\end{align*}
	SI prefixes: $10^{-3}$ milli (m), $10^{-6}$ micro ($\mu$), $10^{-9}$ nano (n), $10^{-12}$ pico (p).
}

For two electrons spaced 1 millimeter apart, we can calculate $F_E=k \left|q_1\right|\left|q_2\right|/r^2$ and $F_G=Gm_1m_2/r^2$. It turns out that $F_E$ is about $4\times 10^{42}$ times greater than $F_G$. So the gravitational force can often be ignored because it is so weak compared to electric force.

Electric force is a vector quantity, whose direction can be indicated with $+/-$ sign or with $\hat i,\hat j,\hat k$.

\remark{Electrostatics vs circuits}{
	In electrostatics, charges are not moving (hence ``static''). In a circuit, which for our purposes is direct current (DC), charges are flowing through the circuit and are not stationary.
}

\subsection{Atomic Structure}

Matter is made of atoms, ions and molecules, which are in turn made of protons (with positive charge), electrons (with negative charge) and neutrons (with no charge).

The mass of an electron is \ul{significantly lower} than the mass of a proton or neutron:
\begin{equation*}
	(m_e=9.11\times 10^{-31}\,\si{\kilo\gram})\ll m_p=m_n=1.67\times 10^{-27}\,\si{\kilo\gram}
\end{equation*}
Charge is quantized: the magnitude of charge on a proton or electron is the elementary charge:
\begin{equation*}
	\left|Q_e\right|=\left|Q_p\right|=1.6\times 10^{-19}\,\si{\coulomb}
\end{equation*}
There is \ul{no} unit of charge smaller than the elementary charge, and all possible charges is a multiple of it:
\begin{equation*}
	\left|Q\right|=n \left( 1.6\times 10^{-19} \right)\si{\coulomb} \qquad n\in\mathbb Z^+
\end{equation*}
\definition{An object being ``charged'' indicates the sign of the \ul{net} charge, which is either positive or negative.
\begin{itemize}
	\item Positively charged: the object is deficient of electrons (has \ul{less} electrons than protons).
	\item Negatively charged: the object is in excess of electrons (has \ul{more} electrons than protons).
\end{itemize}}
The mass of excess electrons in a negatively charged object or the mass of electrons removed in a positively charged object is negligible, because the mass of an electron is so low.

\theorem*{
	Charges on a metal/conductive sphere (solid or hollow) will always be on the surface of the sphere, because charges tend to maximize distance between each other. Thus, the \ul{net charge is always on the surface}.
}

\theorem*{
	In an insulator, there is no free flow of charges. Charges are stationary on the surface of the object.
}

\subsection{Polarization}

\definition{\textbf{Polarization} is a separation of positive and negative charges within an object. This is done by shifting the statistical distribution of electrons to one side of an object (as explained by quantum mechanics, electron orbitals, etc.). It does \ul{not} create charge, and will still have the same net charge as before.}

A charged object can induce polarization in nearby neutral objects. Once the neutral object is polarized, like charges repel and are now further apart, meaning opposite charges are now closer together. Therefore, the two objects will attract.

In conductors, electrons move more readily. Polarization results in a side with excess electrons (negative) and a side with a deficit in electrons (positive).

In insulators, electrons are not free to move, and polarization occurs by changing the shape of electron clouds/orbitals. The electron cloud becomes asymmetrically shaped, and the ``center of electrons/charge'' is not at the atomic nucleus. The individual atoms become polarized.

Polar molecules (have a dipole moment) can orient themselves in an electric field due to attraction.

Polarization explains a number of real-world occurances, such as:
\begin{itemize}
	\item Balloons stick to a wall because the balloon is negative charged (has high affinity for electrons), and the wall molecules polarize.
	\item Water streams bend because water molecules reorient toward a charged object.
	\item Small pieces of paper jump toward a balloon because the negatively charged balloon surface polarizes it.
\end{itemize}

\subsection{Charging}

\textbf{Charging by friction.} Different objects (insulators) have a different tendency to gain or lose net charge when rubbed with another object. Some will gain electrons while some will lose electrons.
\begin{itemize}
	\item Glass and human hair will lose electrons and become more positive.
	\item Nylon and wood will gain electrons and become more negative.
\end{itemize}

\textbf{Charging by conduction.} This can be used for both insulators and conductors. It involves the movement of charges to/from a neutral conductor from/to a charged object by contact.

\textbf{Charging by induction.} 
\begin{enumerate}
	\item Start with a charged object (``rod'') and a neutral conductor (``sphere'').
	\item When the rod is brought \ul{near} (not necessarily contacting) the sphere, free electrons will rush towards/away from the rod. The sphere becomes \ul{polarized.}
	\item Connect the sphere to ground.
	\item Electrons from ground will rush towards the sphere, and displace the positively charged particles.
	\begin{center}
		\ul{OR}
	\end{center}
	Electrons from the sphere will rush towards ground, leaving only positively charged particles.
	\item When the ground wire is removed from the sphere and the rod is moved away, the sphere remains charged.
\end{enumerate}
This can be used to charge the sphere with either positive charge or negative charge. It would depend on the charge of the rod we start with.
\begin{itemize}
	\item Negatively charged rod $\to$ positively charged sphere.
	\item Positively charged rod $\to$ negatively charged sphere.
\end{itemize}

\textbf{Electroscopes.} An electroscope is a device used to detect charge. It consists of a large model knob connected via a stem to two metal leaves. If a charged object is brought near the knob, electrons will move towards/away from the leaves, causing them to move upwards (since the leaves become charged and repeal each other).

\subsection{Field Model}

The field model is related to \ul{non-contact forces}, which are gravity, electric force and magnetic force. The field model describes the alteration of space.

\setlength{\columnsep}{0.6in}
\setlength{\columnseprule}{0.4pt}
\begin{minipage}{\textwidth}
\begin{multicols*}{2}
	\begin{center}
		\ul{\textbf{Gravity}}
	\end{center}

	In a gravitational field, all masses are attracted to the center (of mass $M$) by a force given by the law of universal gravitation:
	\begin{equation*}
	    F_G=\frac{GMm}{r^2}
	\end{equation*}
	The gravitational field strength $g$ is:
	\begin{equation*}
	    g=\frac{GM}{r^2}
	\end{equation*}
	and has units $\si{\meter/\second^2}=\si{\newton/\kilo\gram}$. $g$ is a vector quantity depending on the distance vector $r$ and the location of the test mass. The force exerted on a mass $m$ is:
	\begin{equation*}
	    F=gm \implies \Sigma F=ma
	\end{equation*}
	Field lines all point towards the mass $M$. The field is \ul{radially symmetric} and the field lines point \ul{radially inward}. On the Earth's surface, field lines point down and is uniform at $g=9.81\,\si{\meter/\second^2}$.
	
	\vfill\null\columnbreak

	\begin{center}
		\ul{\textbf{Electricity}}
	\end{center}
	In an electric field, charges are either attracted to or repelled from the central charge ($Q$), given by:
	\begin{equation*}
		F_E=\frac{kQq}{r^2}
	\end{equation*}
	\newline The electric field strength $E$ is:
	\begin{equation*}
		E=\frac{kQ}{r^2}
	\end{equation*}
	and has units $\si{\newton/\coulomb}$. $E$ is a vector quantity depending on the distance vector $r$ and the location of the test charge. The force exerted on a charge $q$ is:
	\begin{equation*}
		\Sigma F=Eq
	\end{equation*}
	Field lines are more complex because there are both attractive and repulsive force. The direction of field lines are the direction of $F_E$ on a \ul{positive charge}. When close to $Q$, the field lines are approximately straight and are uniform.
\end{multicols*}
\end{minipage}

\subsection{Electric Fields}

\definition{A \textbf{capacitor} is made of two plates with opposite charge. If the distance between the places is small, it can be thought of as being uniform.}

\definition{A field is \textbf{uniform} if the electric force on a test charge is the same (in both direction and magnitude) at all points in space. A field is \textbf{nonuniform} if it is not.}

Properties of electric fields:
\begin{itemize}
	\item Electric field is always a vector quantity around the source charge ($Q$). At any location, the direction of the electric field is \ul{tangent} to field lines. Field lines can be curved.

	\item Field lines are \ul{perpendicular} to conducting surfaces.

	\item Field lines point in the direction of a force on a \ul{positive}

	\item The density/spacing of field lines indicate the relative strength of the field. Denser lines/lower distance between lines indicate a stronger electric field. If the strength of the electric field $E=kQ/r^2$ doubles, the number of lines doubles.

	\item Field lines do not cross each other, as this would indicate two different electric field values/forces at the same point (the electric field is not a function).

	\item Electric field strength $E$ is a vector quantity.
\end{itemize}

\theorem{Electric field strength}{
	The strength of an electric field created by a point charge with charge $Q$ from a certain distance $r$ is:
	\begin{equation*}
	    \lVert E \rVert =\frac{k \left|Q\right|}{r^2}
	\end{equation*}
	The units for electric field strength $E$ is either newton per coulomb (\si{\newton\per\coulomb}) or volt per meter (\si{\volt\per\meter}). This is not dependent on the strength of any test charge. The force acting on a test charge of charge $q$ is:
	\begin{equation*}
	    \lVert F_E \rVert =E \left|q\right|=\frac{k \left|Q\right| \left|q\right|}{r^2}
	\end{equation*}
}

\subsection{Electrical Potential and Potential Energy}

In a uniform gravitational field, change in potential energy is:
\begin{equation*}
    \Delta U_g=mg\Delta h
\end{equation*}
where $g$ is the gravitational field strength, which is $9.81\,\si{\meter/\second^2}$ on the surface of the Earth.

\defitheorem{Gravitational potential}{
	\textbf{Gravitational potential} is gravitational potential energy per unit mass:
	\begin{equation*}
		V=\frac{U_g}{m}=\frac{mgh}{m}=gh
	\end{equation*}
	and is measured \si{\meter^2\per\second^2} or \si{\joule\per\kilogram}.
}
This concept can be similarly used for electric potential.
\defitheorem{Electric potential and potential energy in a uniform field}{
	In a uniform electric field (between two charged plates) of strength $E$, the \textbf{electric potential energy} $U_E$ of a charge $q$ at a distance $x$ to the plate of opposite charge from $q$ is:
	\begin{equation*}
	    \Delta U_E=qEx
	\end{equation*}
	\textbf{Electric potential} is potential energy per unit charge:
	\begin{equation*}
	    V=\frac{U_E}{q}=\frac{qEx}{q}=Ex=\frac{kQ}{x}
	\end{equation*}
	Change in potential energy associated with a change in potential is:
	\begin{equation*}
		\left|\Delta U_E\right|=\Delta|q\underbrace{Ex}_{V}|=\Delta \left|qV\right|=\left|q\right|\left|\Delta V\right|
	\end{equation*}
}

\definition{\textbf{Equipotential lines/surfaces} are lines/surfaces on which electric potential is constant. They are described in terms of the electric potential on all points on the line/surface measured in volts (V). They behavior similarly to contours.}

\begin{itemize}
	\item Potential is \ul{highest} when close to a charge, and decreases as distance $r,x$ increases.
	\item Positive point charges have positive potential and negative point charges have negative potential. For both positive and negative charges, electric potential at infinite distance is $0\,\si{\volt}$.
	\item Positive charges move from high to low potential values.
	\item Electric field lines/surfaces are perpendicular to equipotential lines.
	\item Field strength is inversely proportional is inversely proportional to the spacing between field lines.
	\item Electric field is a vector quantity, and electric potential is a scalar quantity.
\end{itemize}

If a point $P$ is affected by multiple charges, the electric potential at $P$ is the sum of the electric potential created by each charge:
\begin{equation*}
    V_i=\frac{k \left|Q_i\right|}{x_i}
	\qquad
	V_P=\sum_i V_i
\end{equation*}

Electric potential is measured in volts (V). From this equation:
\begin{equation*}
	\left[ \si{\volt}=\si{\frac{\newton\per\coulomb}{\meter} }=\si{\frac{\joule\meter}{\coulomb\meter}}=\si{\frac{\joule}{\coulomb}}\right]
\end{equation*}
If the potential difference $\Delta V$ is known, change in electric potential energy $\Delta U_E$ can be found. Kinetic energy $K$ can also be found: since energy is conserved, $\Delta K=-\Delta U_E$.

\subsection{Energy}

In mechanics, the unit for energy is the joule (J), which is defined as:
\begin{equation*}
	\left[ \si{J}=\si{\newton\meter}=\si{\frac{\kilo\gram\meter^2}{\second^2}} \right]
\end{equation*}
Energy can also be measured in electron-volts (eV). The change in energy of a proton or electron (with the same charge $q=1.6\times 10^{-19}\,\si{\coulomb}$) is given by:
\begin{equation*}
    \left|\Delta U_E\right|
	=\left|q\right|\cdot \Delta V
	=1.6\times 10^{-19}\cdot 1
	=1.6\times 10^{-19}\,\si{\joule}
\end{equation*}
\definition{A \textbf{electron-volt} is equal to the kinetic energy gained by an elementary charge (proton or electron) as it accelerates across a potential difference of 1 V.
\begin{equation*}
	1\,\si{\electronvolt}=1.9\times 10^{-19}\,\si{\joule}
\end{equation*}}

The energy of assembly is the energy needed to construct a system of point charges.
\begin{enumerate}
	\item The energy it takes to add the first point charge is zero, because there is zero potential initially and zero potential with only one charge.
	\item The change in energy to add a second charge is:
		\begin{equation*}
		    \Delta E=q\Delta V=q_2 \left( \frac{kq_1}{d}-0 \right)=\frac{kq_1q_2}{d}
		\end{equation*}
		The second charge starts at ``infinity'' so it has a potential of $0$.
	\item The change in energy to add a third chage is:
		\begin{equation*}
		    \Delta E=q_3 \left( \frac{kq_1}{d}+\frac{kq_2}{d}-0 \right)=\frac{kq_3(q_1+q_2)}{d}
		\end{equation*}
\end{enumerate}
The total energy to construct the system is:
\begin{equation*}
	\text{Total Energy of Assembly}=\frac{kq_1q_2}{d}+\frac{kq_1q_3}{d}+\frac{kq_2q_3}{d}
\end{equation*}
\theorem{Electric potential energy of a system}{
	The total potential energy of a system, which, \ul{like gravitational energy}, is the sum of the potential energies generated by each combination of two charges. The actual charges are substituted, not only their magnitudes.
	\begin{equation*}
		U_E=\sum \frac{kq_1q_2}{r}
	\end{equation*}
	Two like charges tend to move apart because they have high potential energy when held together, and moving away reduces potential energy. Two opposite charges attract because they have lower potential energy (negative sign, but high magnitude) when close together.

	Electric potential energy $U_E$ equals the total energy of assembly.
}

\subsection{Electrostatics and Circuits}

When a circuit is open electrons are moving in random directions at $10^6\,\si{\meter/\second}$.

When a circuit is closed, the \ul{net movement} of electrons is in a certain direction (direction of electron flow, opposite of conventional current). There are electrons moving in the opposite direction still. The speed of the net movement (``drift speed'') is around $10^{-5}$ to $10^{-4}$ \si{\meter\per\second}. Individual electrons are much faster though.

\theorem{Current}{
	Current (in Ohm's law as $I=\Delta V/R$) is the net flow of charge per unit time and is measured in amperes (A) or coulombs per second (\si{C/s}):
	\begin{equation*}
	    I=\frac{Q}{t}=\frac{\Delta Q}{\Delta t}=\diff qt
	\end{equation*}
	Current can also be defined as:
	\begin{equation*}
	    I=Nev_dA
	\end{equation*}
	where:
	\begin{align*}
		N&=\text{conduction electron density (number of electron carriers per unit volume)} \\
		e&=\text{elementary charge (\si{\coulomb})} \\
		 &=1.6\times10^{-19}\,\si{\coulomb} \\
		v_d&=\text{drift velocity (\si{\meter/\second})} \\
		A&=\text{cross-sectional area (\si{\m^2})}
	\end{align*}
}

Current $I=\Delta V/R$ is measured in amperes (A): $\left[ \si{A}=\si{\V/\ohm} \right]$.

Power $P=I\Delta V=I^2R=\Delta V^2/R$ is measured in watts (W): $\left[ \si{\watt}=\si{\joule/\second} \right]$.

Resistivity $R=\rho l/A$ is measured in ohms ($\si{\ohm}$): $\left[ \si{\ohm}=\si{\volt/\ampere} \right]$.

\defitheorem{Conductivity}{
	Conductivity $\sigma$ measures a material's ability to conduct electricity:
	\begin{equation*}
	    \sigma=\frac{1}{\rho}
	\end{equation*}
	where:
	\begin{align*}
		\sigma&=\text{Conductivity (\si{1/\ohm\meter})} \\
		\rho&=\text{Resistivity (\si{\ohm\meter})}
	\end{align*}
}

\definition{
	\textbf{Drift velocity} $v_d$ is the average displacement traveled by charge carriers per unit time, and as a measure of velocity is measured in \si{\meter/\second}.
}

\defitheorem{Current density}{
	Current density is the current per unit area and is a vector quantity.
	\begin{align*}
		\vec J
		&=\frac{I}{A}
		\intertext{Current can be expressed as an integral of current density:}
		I
		&=\int \vec J\,\mathrm{d}A
		\stackrel?=\vec JA
		\intertext{Electric field strength is:}
		\vec E&=\rho \vec J
	\end{align*}
	where:
	\begin{align*}
		J&=\text{current density (\si{A/m^2=\si{(C/s)/m^2}})} \\
		I&=\text{current (A)} \\
		A&=\text{cross-sectional area (\si{m^2})} \\
		E&=\text{electric field strength (\si{N/C=V/m})}
	\end{align*}
	$J$ and $E$ are vector quantities (even though from $J=I/A$, it does not appear as such!).
}

\subsection{Continuous Charge Distribution}

An infinite number of infinitesimally small point charges (with charge $\mathrm{d}q$) are distributed throughout an object. We want to determine the net electric field $E_\text{net}$ at a location $P$.

\theorem{$\ll$ and $\gg$}{
	If $a\ll b\iff b\gg a$, then $a$ is small compared to $b$. Then any sum $a+b$ can be considered to be only $b$.
}

\subsubsection{Thin rod, perpendicular point of interest}

We want to find the net electric field at a point $P$ located on the perpendicular bisector of the rod (the line from $P$ to the center of the rod forms a right angle with the rod).

The rod has a total charge $+q$, and $P$ is at a perpendicular distance $y$ from the rod. Every section of the rod, each with charge $\mathrm{d}q$ and with $\mathrm{d}x$, has an effect on the rod. Charges from sections on both sides (both on the left and right) of the rod will cancel out with each other, so the net electric field is \ul{directly away} from the rod.

\defitheorem{Linear charge density}{
	The linear charge density $\lambda$ of the rod is:
	\begin{equation*}
	    \lambda=\frac qL=\diff qx
		\iff
		\mathrm{d}q=\lambda \,\mathrm{d}x
	\end{equation*}
}
From the definition of electric field, the electric field created by the rod section is:
\begin{equation*}
    E=\frac{kQ}{r^2}
	\implies
	\mathrm{d}E=\frac{k \,\mathrm{d}q}{r^2}=\frac{kq \,\mathrm{d}x}{r^2}
\end{equation*}
We only care about the vertical component of the charge on $P$ by each rod section (horizontal components cancel out). The vertical component is:
\begin{equation*}
	\mathrm{d}E_\text{vertical}
	=\mathrm{d}E\cos\theta
	=\frac{kq \cos\theta}{r^2}\,\mathrm{d}x
\end{equation*}
Integrating this over the rod (for each section) gives the net electric field. Distance $r$ can be expressed in terms of $x$ and $y$ components of distance: $r=\sqrt{x^2+y^2}$.
\begin{align*}
	E_\text{net}
	&=\int \mathrm{d}E_\text{vertical}
	=\int \frac{k\lambda\cos\theta}{r^2}\,\mathrm{d}x
	=k\lambda \int \frac{\cos\theta}{r^2}\,\mathrm{d}x
	=k\lambda \int_{-L/2}^{L/2}\frac{1}{(x^2+y^2)^{3/2}}\,\mathrm{d}x \\
	&=\Big[ \text{ ... trig. substitutions ... } \Big]
	=k\lambda y \left[\frac{x}{y^2 \sqrt{x^2+y^2}}\right]_{-L/2}^{L/2}
	=\boxed{\frac{2k\lambda L}{y\sqrt{L^2+4y^2}}}
\end{align*}
where:
\begin{align*}
	k&=\text{Coulomb constant} \\
	\lambda&=\text{linear charge density (uniform/constant)} \\
	y&=\text{distance between rod and $P$} \\
	L&=\text{length of rod}
\end{align*}
\textbf{Special case 1.} If $P$ is very far from the rod, then $y\gg L$ and $L/y\to 0$. Also, recall that $\lambda=q/L \implies q=\lambda L$.
\begin{equation*}
    E
	=\frac{2k\lambda L}{y \sqrt{\cancelto0L^2+4y^2}}
	=\frac{2k\lambda L}{2y^2}
	=\boxed{\frac{kq}{y^2}}
\end{equation*}
\textbf{Special case 2.} If the rod is infinitely long, then $L\gg y$
\begin{equation*}
	E
	=\frac{2k\lambda L}{y \sqrt{L^2+\cancelto0{4y^2}}}
	=\frac{2k\lambda \cancel L}{y\cancel L}
	=\boxed{\frac{2k\lambda}{y}}
\end{equation*}

\theorem{Electric field due to infinitely long rod}{
	The electric field due to an infinitely long rod with charge density $\lambda$ on a point a distance $r$ away is:
	\begin{equation*}
	    E=\frac{2k\lambda}{r}
	\end{equation*}
	Electric field strength $E$ is directly proportional to the inverse of distance $1/r$.
}

\subsubsection{Thin rod, colinear point of interest}

A point $P$ is located on the same line in space as the rod of length $L$ and charge density $\lambda$, at a distance $a$ from the closest end. (If the rod were to be infinite distance, $P$ would be on the rod.)

As before, linear charge density gives $\mathrm{d}q=\lambda \,\mathrm{d}x$ and electric field strength $E=kQ/r^2$.
\begin{equation*}
	\mathrm{d}E
	=\frac{k \,\mathrm{d}q}{r^2}
	=\frac{k\lambda \,\mathrm{d}x}{r^2}
	=\frac{k\lambda}{x^2} \,\mathrm{d}x
\end{equation*}
The total charge is the integral of $\mathrm{d}E$ over the rod, which is bounded by $a\leq x\leq a+L$.
\begin{align*}
    E
	&=\int_{a}^{a+L}\mathrm{d}E
	=\int_{a}^{a+L}\frac{k\lambda}{x^2}\,\mathrm{d}x
	=k\lambda \left[-\frac 1x\right]_{a}^{a+L}
	=-k\lambda \left( \frac{1}{a+L}-\frac{1}{a} \right) \\
	&=-k\lambda\cdot \frac{a-(a+L)}{a(a+L)}
	=k\lambda\cdot \frac{(a+L)-a}{a(a+L)}
	=\boxed{\frac{k\lambda L}{a(a+L)}}
\end{align*}

\textbf{Special case.} If $P$ is infinitely far away from the rod, then $a\gg L$. Recall that $k\lambda L=q$.
\begin{equation*}
    E
	=\frac{k\lambda L}{a(a+\cancelto0L)}
	=\frac{k\lambda L}{a^2}
	=\boxed{\frac{kq}{a^2}}
\end{equation*}
This behaves the same as a point charge (for which $E=kq/r^2$).

\subsubsection{Semicircular arc}

Unlike the previous cases, the semicircular arc is negatively charged with net charge $-q$. It has radius $r$. $P$ is a point at the center of the semicircle. $\mathrm{d}E$ is the electric field on $P$ by a section of the arc with charge $\mathrm{d}q$. The same section on the other side of the arc will cancel out its horizontal component. So, like before, the net electric field on $P$ points straight up.

The charge on the semicircular arc is distributed along the arc as if it were one dimensional. The charge density $\lambda$ is still $q/L$, where $L$ is the length of the arc. $s$ is the arc length, and $\mathrm{d}s$ is the length of the arc section.
\begin{equation*}
    L=\frac12 2\pi r=\pi r
	\implies \lambda=\frac{q}{L}=\frac{q}{\pi r}=\diff qs
	\implies
	\mathrm{d}q
	=\lambda \,\mathrm{d}s
\end{equation*}
Arc length $s=r\theta \implies \mathrm{d}s=r \,\mathrm{d}\theta$. The vertical component of the charge from the section is:
\begin{equation*}
	\mathrm{d}E_\text{vertical}
	=\mathrm{d}E\sin\theta
	=\frac{k \,\mathrm{d}q}{r^2}\sin\theta
	=\frac{k \lambda\sin\theta}{r^2}\,\mathrm{d}s
	=\frac{k \lambda\sin\theta}{r^2}r\,\mathrm{d}\theta
	=\frac{k \lambda\sin\theta}{r}\,\mathrm{d}\theta
\end{equation*}
Integrating this over the arc gives the net electric field. Because this is a semicircular arc, $0\leq \theta\leq \pi$, which are the bounds of integration.
\begin{align*}
	E 
	&=\int_{0}^{\pi}\mathrm{d}E_\text{vertical}
	=\int_{0}^{\pi}\frac{k\lambda \sin\theta}{r}\,\mathrm{d}\theta
	=\frac{k\lambda}{r}\int_{0}^{\pi}\sin\theta\,\mathrm{d}\theta
	=\frac{k\lambda}{r}\left[-\cos\theta\right]_{0}^{\pi} \\
	&=-\frac{k\lambda}{r}\left( \cos\pi-\cos0 \right)
	=-\frac{k\lambda}{r}\left( -1-1 \right)
	=-(-2)\frac{k\lambda}{r}
	=\boxed{\frac{2k\lambda}{r}}
\end{align*}
In general, the electric field of any circular arc is:
\begin{equation*}
	E=\frac{k\lambda}{r}\Big[\sin\theta\Big]_{-\theta}^{\theta}
\end{equation*}
where $[-\theta,\theta]$ are the bounds of the arc.

\subsubsection{Circle of charge}

The effect of a circle of charge on a point lying on the line through the center of the circle, and forming a right angle with the circle. The distance between the center of the circle and the point is $z$ and the circle has radius $R$.

A small section of the circle of charge has charge $\mathrm{d}q$ and arc length $\mathrm{d}s$. $r$ is the distance between the section and the point of interest, forming an angle $\theta$.

Linear charge density of the charge section is:
\begin{equation*}
    \lambda=\frac qL=\diff qs
	\implies \mathrm{d}q=\lambda \,\mathrm{d}s
\end{equation*}
The electric field from the charge section is:
\begin{equation*}
	E=\frac{kq}{r^2}
	\implies
    \mathrm{d}E
	=\frac{k \,\mathrm{d}q}{r^2}
	=\frac{k\lambda \,\mathrm{d}s}{r^2}
\end{equation*}
The vertical components of $\mathrm{d}E$ cancel, so we are only interested in the horizontal component. From the diagram, $r$ is the hypotenuse: $r^2=R^2+z^2$ and $\cos\theta=z/R$.
\begin{equation*}
	\mathrm{d}E_\text{horizontal}
	=\mathrm{d}E\cos\theta
	=\frac{k\lambda \,\mathrm{d}s}{r^2}\cos\theta
	=\frac{k\lambda \,\mathrm{d}s}{z^2+R^2}\cdot \frac{z}{r}
	=\frac{k\lambda \,\mathrm{d}s}{z^2+R^2}\cdot \frac{z}{(z^2+R^2)^{1/2}}
	=\frac{k\lambda z \,\mathrm{d}s}{(z^2+R^2)^{3/2}}
\end{equation*}
The net charge is the integral of $\mathrm{d}E_\text{horizontal}$, but everything is constant except $\mathrm{d}s$. We integrate over the circumference of the circle: $0\leq s\leq 2\pi R$.
\begin{equation*}
    E
	=\int \frac{k\lambda z}{(z^2+R^2)^{3/2}}\,\mathrm{d}s
	=\frac{k\lambda z}{(z^2+R^2)^{3/2}}\int_0^{2\pi R} \mathrm{d}s
	=\frac{k\lambda z}{(z^2+R^2)^{3/2}}\cdot2\pi R
	=\boxed{\frac{kqz}{(z^2+R^2){3/2}}}
\end{equation*}
\textbf{Special case 1.} If $P$ is at the center of the circle of charge, then $z=0\implies E=0$.

\textbf{Special case 2.} If $P$ is very far from the ring, $z\gg R$.
\begin{equation*}
	E
	=\frac{kqz}{(z^2+\cancelto0{R^2})^{3/2}}
	=\frac{kqz}{z^3}
	=\frac{kq}{z^2}
\end{equation*}
When the distance is high compared to the radius of the circle, the ring of charge becomes a point charge.

\subsubsection{Solid sphere or hollow shell}

The sphere/shell has radius $R$.

\textbf{Inside.} When inside the shell, sphere or \ul{any} conductor, the net electric charge is $0$.

\textbf{On the surface.} The net electric field is:
\begin{equation*}
	E=\frac{kQ}{R^2}
\end{equation*}

\textbf{Away from the surface}. The net electric field is:
\begin{equation*}
	E=\frac{kQ}{r^2}
\end{equation*}
where $r$ is the distance from the point of interest to the center of the sphere/shell.

When on/away from the surface, entire sphere/shell can be thought of as being \ul{collapsed into a point charge}.

\subsection{Gauss's Law}

We can consider electric field lines/vectors as a flowing fluid (such as a gas or liquid).

\definition{
	An \textbf{area vector} is a vector representing a surface:
	\begin{itemize}
		\item Its length is proportional to the magnitude of the area.
		\item Perpendicular to the surface area it represents (assuming the surface is small enough to be a plane).
		\item Since electric field $E$ is perpendicular to a conducting surface, $E$ at a point is a \ul{scalar multiple} of $\mathrm{d}A$.
	\end{itemize}
}
\defitheorem{Flux}{
	\textbf{Flux} ($\Phi$) is the flow of a fluid through a cross-sectional area (CSA, A).
	\begin{equation*}
	    \Phi=\mathbf v\cdot \mathbf A=vA\cos\theta
	\end{equation*}
	where:
	\begin{align*}
		\mathbf v&=\text{velocity of liquid} \\
		A&=\text{cross-sectional area vector} \\
		v=\lVert \mathbf v \rVert &=\text{speed of liquid (\si{m/s})} \\
		A=\lVert \mathbf A \rVert &=\text{cross-sectional area (\si{m})} \\
		\theta&=\text{angle between flow vector and area vector (\si{\radian})}
	\end{align*}
	Flux is the volume flow rate and is measured in \si{m^3/s}. Flux is a scalar.
}

\defitheorem{Electric flux}{
	\textbf{Electric flux} is the amount of electric field passing through a surface area. Electric flux is given by:
	\begin{equation*}
		\Phi_E=\mathbf E\cdot\mathbf A=EA\cos\theta
	\end{equation*}
	where:
	\begin{align*}
		\mathbf E&=\text{electric field} \\
		A&=\text{cross-sectional area vector} \\
		E=\lVert \mathbf E \rVert &=\text{strength/magnitude of electric field (\si{N/C}, \si{V/m})} \\
		A=\lVert \mathbf A \rVert &=\text{cross-sectional area (\si{m})} \\
		\theta&=\text{angle between electric field vector and area vector (\si{\radian})}
	\end{align*}
	Suppose we know the electric field $\mathrm{d}\mathbf E$ created by a small section of a surface with area vector $\mathrm{d}\mathbf{A}$.
	\begin{align*}
		\mathrm{d}E
		&=\mathbf E\cdot \mathrm{d}\mathbf A
		\intertext{We integrate the electric field over the surface for the total electric field. We integrate over a surface, so we use the \textbf{surface integral}.}
		E
		&=\oint \mathbf E\cdot\mathrm{d}\mathbf A
	\end{align*}
	If $Q_\text{enc}=0$, electric flux $\Phi_E=0$ even if electric fields puncture the surface. Charges outside the Gaussian surface have an equal number of field lines that puncture into and out of the surface.
}

\definition{
	A \textbf{Gaussian surface} is a \ul{closed} surface.
}

We find an equation for flux as a function of the charge enclosed in a Gaussian surface. Consider a point charge:

\begin{enumerate}
	\item Construct a spherical Gaussian surface of radius $r$ around the charge $+q$.
	\item Consider a small surface on the sphere with area vector $\mathrm{d}\mathbf A$. Let $\mathrm d\mathbf E$ be the electric field at the surface.
	\item Using equations for electric flux:
		\begin{equation*}
		    \Phi_E
			=EA\cos\theta
			=\frac{kQ}{\cancel{r^2}}(4\pi \cancel{r^2})
			=\frac{1}{\cancel{4\pi}\varepsilon_0}(\cancel{4\pi} Q)
			=\boxed{\frac{Q_\text{enc}}{\varepsilon_0}}
		\end{equation*}
		``enc'' represents the charge enclosed by the Gaussian surface.
\end{enumerate}
\theorem{Point charge in Gaussian surface}{
	The electric flux of a charge $Q_\text{enc}$ enclosed in \ul{any} Gaussian surface is:
	\begin{equation*}
		\Phi_E=\frac{Q_\text{enc}}{\epsilon_0}
	\end{equation*}
	where:
	\begin{align*}
		\Phi_E&=\text{electric flux} \\
		Q_\text{enc}&=\text{enclosed charge (C)} \\
		\varepsilon_0&=\text{vacuum permittivity (\si{C^2/Nm^2})} \\
		&=8.85\times 10^{-12}\,\si{C^2/Nm^2}
	\end{align*}
}

\defitheorem{Surface and volume charge density}{
	\begin{equation*}
	    \sigma=\frac{Q}{A} \qquad
	    \rho=\frac{Q}{V}
	\end{equation*}
	where:
	\begin{align*}
		Q&=\text{total charge (C)} \\
		A&=\text{area (\si{m^2})} \\
		V&=\text{volume (\si{m^3})} \\
		\sigma&=\text{surface charge density (\si{C/m^2})} \\
		\rho&=\text{volume charge density (\si{C/m^3})}
	\end{align*}
}

Electric field can be found by solving for $E$ in:
\begin{equation*}
	\Phi_E=\frac{Q_\text{enc}}{\varepsilon_0}=EA\cos\theta
\end{equation*}

\subsubsection{Thin rod}

\subsubsection{Cylinder}

\subsubsection{Plane}

\subsubsection{Capacitor}

A capacitor is composed of two conducting and charged planes very close to each other. Net charge is inside the capacitor between the two planes.

We choose a cylinder starting on one plane and ending in between the two planes as the Gaussian surface. Consider one touching the left plane. For the right end cap, area vector is in the same direction as electric field:
\begin{equation*}
    \Phi_E=EA\cos\theta=EA
\end{equation*}
At the left end cap, electric field is zero because it is inside the plane. Electric field inside a conductor is zero.
\begin{equation*}
    \Phi_E=\cancelto0EA\cos\theta=0
\end{equation*}
In the sheath, $E$ and $A$ are orthogonal:
\begin{equation*}
	\Phi_E=EA\cancelto0{cos\theta}=0
\end{equation*}
The total flux for the Gaussian surface is $\Phi_E=EA$. Surface charge density $\sigma$ is $Q_\text{enc}$ divided by the area of the end cap: $\sigma=Q_\text{enc}/A$. Equating this with $\Phi_E=Q_\text{enc}/\varepsilon_0$:
\begin{equation*}
	\Phi_E=\frac{Q_\text{enc}}{\varepsilon_0}=EA
	\implies
	E=\frac{Q_\text{enc}}{A\varepsilon}=\boxed{\frac{\sigma}{\varepsilon_0}}
\end{equation*}
When outside the capacitor, choose a Gaussian surface as a surface partially outside the capacitor and intersecting one plate. $E=0$ both inside the conducting plane and outside the capacitor, so the total flux $\Phi_E=0\implies E=0$.

\subsubsection{Insulting sphere}

A charge is uniformly distributed throughout an insulating sphere of radius $R$.

\textbf{Inside the sphere.} To consider electric field inside the sphere, use a smaller sphere with radius $r$ as the Gaussian surface. At all points on the smaller sphere, the electric field points out normal to its surface. $\mathrm{d}A$ is always in the same direction as the electric field.
\begin{equation*}
    \Phi_E=EA\cos\theta=EA
\end{equation*}
The portion of charge enclosed in the Gaussian sphere is $r^3/R^3$.
\begin{equation*}
	\Phi_E=\frac{Q_\text{enc}}{\varepsilon_0}=\frac{r^3Q}{R^3\varepsilon_0}
\end{equation*}
$A$ is the surface area of the sphere: $A=4\pi r^2$. Combining these:
\begin{equation*}
    EA=\frac{r^3Q}{R^3\varepsilon_0}
	\implies
	E=\frac{r^3Q}{(R^3\varepsilon_0)(4\pi r^2)}
	=\boxed{\frac{Qr}{4\pi R^3\varepsilon_0}
	=\frac{kQr}{R^3}}
\end{equation*}
The net electric field inside the sphere is directly proportional to the radius of the smaller sphere. When $r=R$:
\begin{equation*}
    E=\frac{kQR}{R^3}
	=\frac{kQ}{R^2}
\end{equation*}
is the electric field due to a point charge.

\textbf{Outside the sphere.} Choose a larger sphere of radius $r$ as the Gaussian surface. As before, unit vector is in the same direction as the electric field. $A$ is the surface area of the Gaussian sphere: $A=4\pi r^2$.
\begin{equation*}
	\left\{\begin{aligned}
		\Phi_E&=EA\cos\theta=EA \\
		\Phi_E&=\frac{Q_\text{enc}}{\varepsilon_0}=\frac{Q}{\varepsilon_0}
	\end{aligned}\right\}
	\implies E=\frac{Q}{4\pi r^2\varepsilon_0}=\boxed{\frac{kQ}{r^2}}
\end{equation*}

\subsubsection{Shell with thickness}

The shell has inner radius $R_A$ and outer radius $R_B$.

The shell is initially neutral. A charge $+q$ is added to the center of the shell (in the cavity); the inside of the shell becomes negatively charged, and since the shell remains neutral, the outside becomes positively charged.

In all cases, $\mathrm{d}A$ is parallel to $E$: $\cos\theta=1$.

\textbf{Inside the cavity ($0<r<R_A$).} 
Only the point charge $+q$ is enclosed: $Q_\text{enc}=q$.
\begin{equation*}
	\left\{\begin{aligned}
		\Phi_E&=EA\cos\theta=EA \\
		\Phi_E&=\frac{Q_\text{enc}}{\varepsilon_0}=\frac{q}{\varepsilon_0}
	\end{aligned}\right\}
	\implies E=\frac{q}{4\pi r^2\varepsilon_0}=\boxed{\frac{kq}{r^2}}
\end{equation*}

\textbf{Inside the surface ($R_A<r<R_B$).}
Negative charges are located at the inner surface ($R_A$), with magnitude equal to $q$. So the enclosed charge $Q_\text{enc}=0$.
\begin{equation*}
	\left\{\begin{aligned}
		\Phi_E&=EA\cos\theta=EA \\
		\Phi_E&=\frac{Q_\text{enc}}{\varepsilon_0}=0
	\end{aligned}\right\}
	\implies E=\boxed{0}
\end{equation*}

\textbf{Outside the shell ($r\geq R_B$).}
The point charge and the net neutrally charged shell is enclosed. The charge of the shell is $0$. So the total enclosed charge is $Q_\text{enc}=q$.
\begin{equation*}
	E=\frac{Q}{4\pi r^2\varepsilon_0}=\boxed{\frac{kq}{r^2}}
\end{equation*}

\textbf{Graph of $E$ against $r$.} $E$ is proportional to $1/r^2$ at all points except when $R_A<r<R_B$, where $E=0$.
\begin{equation*}
    E=\begin{cases}
		0 & R_A<r<R_B \\
		kq/r^2 & 0<r<R_A \text{ or } r>R_B
    \end{cases}
\end{equation*}

\subsection{Irregularly Shaped Objects in Electrostatic Equilibrium}

When an object has a ``weird'' shape, charges are not distributed uniformly on it. Electrons want to be far apart from each other.
\begin{itemize}
	\item When the radius of curvature is large (\textit{The Lord Amar of Hightstowne} reference), electrons are able to move a higher distance apart, resulting in \ul{lower} surface charge density $\sigma$ and therefore a \ul{weaker} electric field $E$.
	\item When the radius of curvature is low, electrons are \ul{not} able to move a higher distance apart (there is no component of force pushing them along the surface, and the charge cannot leave the surface), resulting in \ul{higher} surface charge density $\sigma$ and therefore a \ul{stronger} electric field $E$.
\end{itemize}

\subsection{Electrostatic Equilibrium}

\definition{
	An object is in \textbf{electrostatic equilibrium} if it has a net charge, but the charges on it are not moving.
}
When an object is in electrostatic equilibrium, there are no charges moving in or on the surface. There is no net electric field inside the surface: $E=0$. When inside or on the surface, electric potential is a constant and potential difference at all points is $\Delta V=0$.

\theorem{Net charge}{
	\begin{itemize}
		\item All net charge reside on the surface, as a result of Coulomb's law.
		\item Electric field ``just outside'' the surface is perpendicular to the surface.
	\end{itemize}
}

\subsection{Electric Field, Charge and Potential}

\theorem{Electric field close to conducting surface}{
	Electric field at a location close to a conducting surface with surface charge density $\sigma$ is:
	\begin{equation*}
	    E=\frac{\sigma}{\varepsilon_0}
	\end{equation*}
	where:
	\begin{align*}
		E&=\text{electric field (\si{N/C})} \\
		\sigma&=\text{surface charge density (\si{C/m^2})} \\
		\varepsilon_0&=\text{permittivity of free space (\si{\coulomb^2/\newton\meter^2})} \\
		&=8.85\times 10^{-12}\,\si{\coulomb^2/\newton\meter^2}
	\end{align*}
}

\theorem{Net charge and potential}{
	Net charge moves from an area of high potential to an area of low potential.
	\begin{equation*}
		V_1=\frac{kQ_1}{r_1} \qquad
		V_2=\frac{kQ_2}{r_2} \qquad
		V_1>V_2
	\end{equation*}
	Net charge will flow from area 1 to area 2 until potentials are equal ($V_1=V_2$):
	\begin{equation*}
		\frac{Q_1}{r_1}
		=\frac{Q_2}{r_2}
	\end{equation*}
}

\theorem{Potential and electric field}{
	\begin{equation*}
	    E=\diff Vx \qquad \Delta V=\int E\,\mathrm{d}x
	\end{equation*}
	where:
	\begin{align*}
		E&=\text{electric field (\si{N/C=V/m})} \\
		V&=\text{electric potential (\si{V})}
	\end{align*}
}

\end{document}

